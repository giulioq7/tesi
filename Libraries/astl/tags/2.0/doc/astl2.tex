\documentclass[10pt]{article}
\usepackage{a4wide,epsf}
\usepackage[latin1]{inputenc}
\usepackage[dvips]{color}
\usepackage{longtable}
%\usepackage{graphics}

% Power set
\newcommand{\pset}[1]{P(#1)}
% Cardinal
\newcommand{\card}[1]{|#1|}
% delta1 and delta2
\newcommand{\dun}{\ensuremath{\delta_{1}} }
\newcommand{\ddeux}{\ensuremath{\delta_{2}} }
% right context
\newcommand{\rcontext}{\ensuremath{\vec{c}}}

\begin{document}
\setlength\arrayrulewidth{0.03cm}
\title{{\bf The Automaton Standard Template Library} \\ {\bf ASTL
        version 2.0} \\ [4ex] Reference Documentation}
\author{Vincent Le Maout}
\maketitle
\newpage
\tableofcontents
\newpage
\section{Introduction}
The Automaton Standard Template Library (ASTL) is a set of generic C++
components for efficient automata manipulation. As any library geared
toward supporting the 
generic programming paradigm, it is made of two distinct parts : a
collection of concepts specific to the automata domain which is
described by 
this documentation and a set of software components (containers,
accessors and algorithms) implementing the concepts. 
\subsection{License}
ASTL is open-source, free software. You can redistribute it and/or
modify it under the terms of the GNU Lesser General Public
License as published by the Free Software Foundation; either
version 2.1 of the License, or (at your option) any later version. \\

This library is distributed in the hope that it will be useful,
but WITHOUT ANY WARRANTY; without even the implied warranty of
MERCHANTABILITY or FITNESS FOR A PARTICULAR PURPOSE. For more details,
check the URL \verb+http://www.fsf.org/+ or read the file
\verb+LICENSE.txt+ containing the GNU Lesser General 
Public License version 2.1. This file can be found at the root
directory of the library.
\subsection{Acknowledgments}
I would like to thank all the people who have worked on and with the
library especially Dominique Revuz for his support and advices on the
code design, Xavier Daragon for writing the very first version of this
documentation and Arnaud Adan for his review and extension to the
weighted automata and transducers. 
\subsection{Availability}
The package and this documentation are downloadable from
\verb+http://astl.sourceforge.net/+ which is the official site for
news, announcements and releases.
\subsection{Supported Compilers}
ASTL 2.0 has been successfully used with GNU g++ 2.96 and later.
% the following compilers :
% \begin{itemize}
% \item GNU g++ 2.96 and later.
% \item Borland C++ 5.5.
% \item Microsoft Visual C++ 6.0 Service Pack 5 with a few minor restrictions.
% \end{itemize}
\subsection{Compiling}
ASTL is made of header files, it does not require to compile
any code prior to using it and therefore requires no specific file to
link to. \\
There is a number of \verb+#defines+ modifying the compiler behavior :
\begin{itemize}
\item \verb+ASTL_USES_NAMESPACE+
\end{itemize}
The only information needed is the \verb+include+ subdirectory
location, for instance: \\
\begin{verbatim}
g++ -I/home/vince/astl/include main.cc
\end{verbatim}
\subsection{Files Hierarchy}
ASTL 2.0 is made of five subdirectories :
\begin{enumerate}
\item \verb+bin+ : command line executables
\item \verb+doc+ : LaTeX documentation source, postcript and PDF
  documentation 
\item \verb+include+ : headers 
\item \verb+src+ : source code for the executables
\item \verb+templates+ : code templates
\item \verb+ext+ : some rather experimental extensions to the library
\end{enumerate}
\section{Definitions and Notations}
\subsection{Finite Automaton}
To make our concepts sufficiently generic to satisfy a broad range of
algorithmic constraints, we add to the classical automaton definition
a set of {\em tags}, that is, any data associated to a state and
needed to apply an algorithm. We will however omit tag-related
considerations whenever they are not relevant. \\

\noindent Let $A(\Sigma, Q, I, F, \Delta, T, \tau)$ be a 7-tuple of finite sets
defined as follows :
\begin{center}
\begin{tabular} {ll}
  $\Sigma$  & An alphabet \\
  $Q$       & A set of states \\
  $I \subseteq Q$ & A set of initial states \\
  $F \subseteq Q$ & A set of final states (also called terminal or
  accepting states) \\
  $\Delta \subseteq (Q \times \Sigma \cup \{ \epsilon \} \times Q)$ &
  A set of transitions \\
  $T$       & A set of tags \\
  $\tau \subset (Q \times T)$ & A set mapping a state to its
  associated tag  \\
\end{tabular} \\ [2ex]
\end{center}
We distinguish one special state noted 0 and called the {\em null} or
{\em sink state}. The {\em label} of a transition $(q, \sigma, p) \in
\Delta$ is
the letter $\sigma$, $q$ is the {\em source} state and $p$ is the {\em
        destination} state or {\em aim}. When $\sigma = \epsilon$ (the empty
word) the transition is said to be an {\em $\epsilon$-transition}. \\
[1ex] 
We call {\em incoming} transitions (respectively {\em outgoing}
transitions) of a state $s$, the set of transitions
$(q, \sigma, p) \in \Delta$ such that $p = s$ (respectively $q =
s$). By default, the transitions of a state are its outgoing
transitions. \\ [1ex] 
We will write $\pset{X}$ for the powerset of a set $X$ and $\card{X}$
for its number of elements. \\ [2ex]
To access $\Delta$ we define two transition functions \dun and \ddeux : \\
  $$\dun \: : \: Q \times \Sigma \cup \{ \epsilon \} \rightarrow \pset{Q} $$ 
  $$\ddeux \: : \: Q \rightarrow \pset{\Sigma \times \pset{Q}} $$ \\
\dun retrieves the set of transitions targets given the source state
and a letter. \ddeux allows to access the set of all the outgoing
transitions of a given state. \\ [1ex]
\dun can be naturally extended to words : \\ 
\begin{center}
\begin{tabular}{rcl}
$ \dun ^{*} \: : Q \times \Sigma^{*}$ & $\rightarrow$ & $\pset{Q} $ \\
$ (q, \epsilon)$ & $\mapsto$ & $q $ \\
$ (q, w \cdot a)$ & $\mapsto$ & $\dun (\dun ^{*}(q,w), a) $ \\
\end{tabular} \\ [3ex]
\end{center}
The {\em right context} of a state $q$ is the set of letters labelling
the outgoing transitions of $q$ : $\rcontext(q) = \{ \sigma \in
\Sigma \mid \exists p \in Q, \: (q, \sigma, p) \in \Delta \}$. \\
A {\em path} is a sequence $c = t_{1} \: t_{2} \: ... \: t_{n}$ of
transitions 
$t_{i} = (q_{i}, \sigma_{i}, p_{i})$ such that $\forall i, t_{i} \in \Delta$ and
for $i < n$, $q_{i+1} = p_{i}$. The path length $n$ is noted $|c|$ and
its label is the concatenation of the transitions letters : $w =
\sigma_{1} \: \sigma_{2} \: ... \: \sigma_{n}$. \\ [1ex]
The language recognized by an automaton $A$ is defined by :
$$ L(A) = \{ w \in \Sigma ^{*} \mid \dun ^{*}(I, w) \cap F \not=
\emptyset \} $$
that is, the labels of the paths leading from an initial state to a
final state. \\

\noindent An automaton is said to be {\em deterministic} iff $I$ is a singleton
and there is at most one transition per state which is labeled by a
given alphabet letter, that is, $\card{\dun(q)} \leq 1$, $\forall q \in
Q$. In this case, \dun is defined as :
$$ \dun(q, \sigma) = \left\{ \begin{array}{ll}
p & \mbox{if $(q, \sigma, p) \in \Delta$} \\
0 & \mbox{otherwise}
\end{array} \right. $$
The sink state acts as failure value for the transition function. \\

\paragraph{Example (figure \ref{fa_example})}
\begin{figure}
\begin{center}
\epsfxsize=6cm
\epsfbox{fa_example.eps}
\caption{Example of NFA}
\label{fa_example}
\end{center}
\end{figure}
$A$ is a non-deterministic automaton with $\Sigma = \{ a, b, c \}$, $Q =
\{ 1,2,3,4,5 \}$, $I = \{ 1 \}$, $F = \{ 2, 5 \}$ and $\Delta = \{
(1,a,2), (2,b,2), (1,b,3), (3,a,5), (1,b,4), (4,c,5) \}$. \\
$L(A) = \{ ab^{*}, ba, bc \}$.
\subsection{Containers \& Cursors}
\section{Getting Started}
This section contains code example ranging from basic automaton
operations to cursors manipulation. They cover all aspect of the
library functionnalities and aim at providing an insight of what can
be done and how\footnote{The code examples make no use of the ASTL
  name space so {\tt ASTL\_USES\_NAMESPACE} should not be defined to
  compile them as is.}. 
\subsection{Declaring a Container}
The automaton containers are class templates parameterized by two
types: the alphabet traits and 
the tag type. By default, the alphabet trait is \verb+plain+ (8 bits
\verb+char+) and the tag type is \verb+empty_tag+ (no tags
needed). Also a predefined alphabet traits called \verb+range+ is
provided. It must be instanciated with a builtin integral type
\verb+T+ followed by two constants \verb+x+ and \verb+y+ of type
\verb+T+ defining a subset
\verb+[x, y]+ of the domain \verb+T+. For example,  
\verb+plain+ is defined as follow : 
\begin{verbatim}
typedef range<char, (char) -128, (char) 127> plain;
\end{verbatim}
\subsubsection{DFA}
ASTL provides eight DFA containers sharing the same
interface. They have homogeneous interfaces and behaviors but
heterogeneous implementations and complexities allowing to choose at
utilization time which one fits best the situation. The following
piece of code declares a variety of containers :
\begin{verbatim}
#include <astl.h>
#include <dfa.h>
#include <string>

int main() 
{
  DFA_map<> A1; 
  DFA_matrix<range<char, 'A', 'Z'> > A2;
  DFA_tr<range<int, 0, 1023> > A3;
  DFA_mtf<french, std::string> A4;
  DFA_hash<std::char_traits<char> > A5;
  DFA_bin<ASCII, int> A6;
}
\end{verbatim}
The declaration of \verb+A1+ uses the template default parameters, it
is equivalent to :
\begin{verbatim}
DFA_map<plain, empty_tag> A1;
\end{verbatim}
\verb+A5+ uses the standard character traits \verb+std::char_traits+
as alphabet traits. It can be used to instanciate all containers
except \verb+DFA_matrix+ and must only be used with builtin types
alphabets. 
\subsubsection{Compact DFA}
A compact automaton is a container adapter parameterized by the
adapted automaton type. It is constructed by copy :
\begin{verbatim}
#include <astl.h>
#include <dfa_mtf.h>
#include <dfa_compact.h>

int main()
{
  DFA_mtf<> A;
  DFA_compact<DFA_mtf<> > C(A);
}
\end{verbatim}
The constructor of \verb+C+ builds a copy of \verb+A+ in a compact
representation.
% \subsubsection{Minimal Dynamic Acyclic DFA}
% The minimal dynamic acyclic automaton is parameterized by two
% allocator types (one for states and one for transitions) which can be
% ommited if no specific allocation scheme is required. It does not
% support tags and the alphabet type is necessarily \verb+char+ :
% \begin{verbatim}
% #include <astl.h>
% #include <dfa_min.h>

% int main()
% {
%   DFA_min<> A; 
% }
% \end{verbatim}
\subsubsection{NFA}
\begin{verbatim}
#include <astl.h>
#include <nfa_mmap.h>

int main()
{
  NFA_mmap<plain, int> A;
}
\end{verbatim}
\subsection{Constructing an Automaton}
\begin{verbatim}
#include <astl.h>
#include <dfa.h>
#include <vector>
#include <iterator>

int main()
{
  using namespace std;
  DFA_matrix<> A;
  DFA_matrix<>::state_type q = A.new_state();
  DFA_matrix<>::state_type p = A.new_state();
  A.set_trans(q, 'a', p);
  A.initial(q);
  A.final(p) = true;
  DFA_matrix<>::state_type Q1[10];
  A.new_state(10, Q1);
  vector<DFA_matrix<>::state_type> Q2(10);
  A.new_state(10, Q2.begin());
  vector<DFA_matrix<>::state_type> Q3;
  A.new_state(10, back_inserter(Q3));
}
\end{verbatim}
\subsection{Accessing States Transitions}
\begin{verbatim}
#include <astl.h>
#include <dfa.h>
#include <iostream>

int main()
{
  using namespace std;
  typedef DFA_matrix<> DFA;
  DFA A;
  // Construction...
  DFA::state_type q = A.initial();
  DFA::state_type p = A.delta1(q, 'a');
  if (p == A.null_state) cout << "undefined" << endl;
  DFA::edges_type e = A.delta2(q);
  if (e.find('b') == e.end()) cout << "undefined" << endl;
  DFA::edges_type::const_iterator i;
  for(i = e.begin(); i != e.end(); ++i)
    cout << " source " << q 
         << " letter " << i->first 
         << " aim "    << i->second;
}
\end{verbatim}
\subsection{Minimizing Acyclic DFAs}
\begin{verbatim}
#include <astl.h>
#include <dfa.h>
#include <minimize.h>

int main()
{
  DFA_matrix<plain, minimization_tag> A;
  // Construction...
  acyclic_minimization(A);
}
\end{verbatim}
\subsection{Matching}
\begin{verbatim}
#include <astl.h>
#include <dfa.h>
#include <cursor.h>
#include <language.h>
#include <string>
#include <iostream>

int main()
{
  using namespace std;
  DFA_matrix<> A;
  string w = "word";
  // Construction...
  cursor<DFA_matrix<> > c(A);

  string::const_iterator i = w.begin();
  for(c = A.initial(); i != w.end() && c.forward(*i); ++i);
  if (i == w.end() && c.src_final()) cout << "recognized";

  if (is_in(w.begin(), w.end(), plainc(A))) cout << "recognized too";

  if (is_in(istream_iterator<char>(cin), istream_iterator<char>(),
            plainc(A))) cout << "found on stdin"; 
}
\end{verbatim}
\subsection{Using a Forward Cursor}
\begin{verbatim}
#include <astl.h>
#include <dfa.h>
#include <language.h>
#include <cursor.h>
#include <iostream>

int main()
{
  DFA_matrix<> A;
  // Construction...
  forward_cursor<DFA_matrix<> > c(A, A.initial());
  if (c.first()) 
    do 
      std::cout << " source " << c.src() 
                << " letter " << c.letter() 
                << " aim "    << c.aim();
    while (c.next());


  const char *w = "word";
  if (is_in(w, w + 4, forwardc(A))) std::cout << "recognized";

}
\end{verbatim}
\subsection{Using a Depth-First Cursor}
\subsection{Language Extraction}
\begin{verbatim}
#include <astl.h>
#include <dfa.h>
#include <language.h>
#include <cursor.h>
#include <iostream>

int main()
{
  DFA_matrix<> A;
  // Construction...
  language(std::cout, dfirstc(A));
}
\end{verbatim}
\subsection{Copying}
\begin{verbatim}
#include <astl.h>
#include <dfa.h>
#include <language.h>
#include <cursor.h>
#include <ccopy.h>
#include <iostream>

int main()
{
  DFA_matrix<> A, B;
  // Construction...
  DFA_matrix<>::state_type i = ccopy(A, dfirstc(B));
  A.initial(i);

  DFA_matrix<> C, D;
  // Construction...
  C.initial(clone(C, dfirstc(D)));
}
\end{verbatim}
\subsection{Streams Input/Output}
\begin{verbatim}
#include <astl.h>
#include <dfa.h>
#include <ccopy.h>
#include <stream.h>

int main()
{
  DFA_matrix<> A;
  // Construction...
  dump(cout, dfirstc(A)); // don't save tags
  full_dump(cout, dfirstc(A)); // save tags

  DFA_matrix<> B;
  restore(B, cin);
}
\end{verbatim}
\subsection{On-the-fly Processing}
\begin{verbatim}
#include <astl.h>
#include <dfa.h>
#include <language.h>
#include <set_operation.h>
#include <cursor.h>
#include <string>
#include <iostream>

int main()
{
  DFA_matrix<> A, B;
  // Constructions
  std::string w = "word";
  if (is_in(w.begin(), w.end(), 
            intersectionc(forwardc(A), forwardc(B))))
     std::cout << "recognized";
}
\end{verbatim}
\subsection{By-copy Processing}
\begin{verbatim}
#include <astl.h>
#include <dfa.h>
#include <language.h>
#include <set_operation.h>
#include <cursor.h>
#include <neighbor.h>

int main()
{
  DFA_matrix<> A, B;
  // Constructions
  ccopy(A, dfirstc(neighborc(forwardc(B), "word", 2)));

  DFA_matrix<> C;
  clone(C, dfirstc(notc(diffc(forwardc(A), forwardc(B)))));
}
\end{verbatim}
\subsection{Lazy-construction Processing}
\begin{verbatim}
#include <astl.h>
#include <dfa.h>
#include <lazy.h>
#include <regexp.h>
#include <language.h>
#include <iostream>

int main()
{
  regexp_cursor e("a|b*");
  const char *w = "aaabb";
  if (is_in(w, w + 5, lazyc(e))) std::cout "recognized";

  lazy_cursor<regexp_cursor> c(e);
  if (is_in(w, w + 5, c)) std::cout "recognized too";
}
\end{verbatim}
\subsection{Virtual-traversal Processing}
\begin{verbatim}
#include <astl.h>
#include <dfa.h>
#include <stream.h>
#include <iostream>

// save the automaton:

int main()
{
  DFA_matrix<> A;
  // Construction...
  dump(cout, dfirstc(A));
}

// cut .......................................
\end{verbatim}

\begin{verbatim}
#include <astl.h>
#include <dfa.h>
#include <ccopy.h>
#include <language.h>
#include <iostream>

// extract language:

int main()
{
  clone_cursor<plain> c(std::cin);
  language(std::cout, c);
}
\end{verbatim}
\subsection{Determinizing}
\begin{verbatim}
#include <astl.h>
#include <dfa.h>
#include <nfa.h>
#include <determinize.h>
#include <ccopy.h>

int main()
{
  NFA_mmap<> N;
  // Construction...
  DFA_matrix<> A;
  A.initial(clone(A, dfirstc(forwarddc(N))));
}
\end{verbatim}
\subsection{Displaying and Printing}
\begin{verbatim}
#include <astl.h>
#include <dfa.h>
#include <dot.h>
#include <ccopy.h>
#include <iostream>

int main()
{
  DFA_map<> A, B;
  // Construction...
  dot(std::cout, dfirstc(A)); // don't write tags
  full_dot(std::cout, dfirstc(A)); // write tags
}
\end{verbatim}
\section{Coding Standards}
\subsection{Namespace}
ASTL components may optionally be enclosed in a namespace \verb+astl+
by defining the symbol \verb+ASTL_USES_NAMESPACE+.
\subsection{Exceptions \& RTTI}
So far, ASTL makes no use of the C++ exceptions mechanism nor RTTI
(RunTime Type Information). They can be
safely turned off on the compiler command line
({\tt g++ -fno-exceptions -fno-rtti}).
\subsection{Types and Functions Naming}
ASTL follows the C++ standard way for types naming:
types and functions names contain only lower-case letters and compound
words contain underscores separating components. Most of the time,
words are used literaly without any abbreviation. Examples:
\verb+forward_cursor+, \verb+acyclic_minimization+. \\
For formal template parameters, upper-case letters are used and no
underscore appears, thus minimizing the probability for
names collision between real types and formal parameters. For instance,
the definition:
\begin{verbatim}
template <typename ForwardCursor>
class A
{ };
\end{verbatim}
should avoid any
confusion between the symbol \verb+ForwardCursor+ and the existing type
\verb+forward_cursor+.
% \subsection{Algorithms}
\subsection{Helper Functions}
Whenever it is possible and useful, a helper function is provided to
make component building and initializing easier.
\subsection{Testing \& Debugging}
\verb+src/check_dfa.cc+ \\
\verb+src/check_nfa.cc+ \\
\verb+src/check_cursor.cc+ \\
\verb+check.h+ (coverage test) \\
\verb+debug.h+ debug cursor and trace cursor \\
\section{Concepts}
\subsection{Alphabet}
\subsection{Edges}
\subsection{Container}
\subsection{Cursor}
\section{Models}
\subsection{Alphabets}
\subsection{Containers}
\subsection{Cursors}
%\small
%
%%%%%%%%%%%%%%%%%%%%%%%%%%%%%%%%%%%%%%%%%%%%%%%%%%%%%%%%%%%%%%%%%%%%%%%%
\newpage
\noindent {\huge {\bf cursor$<$DFA$>$}} \\
\noindent \begin{tabular}{p{7.25cm}p{7.25cm}} \hline
\flushleft {\bf Category : {\Large cursors}} & \flushright 
{\bf Component Type : {\Large Type}} 
\end{tabular} \\
\paragraph{Description \\}
A cursor is a pointer to an automaton state that is able to move along
defined transitions. Its purpose is to implement simple traversals
testing if a word is in the language recognized by an automaton.
\paragraph{Example}
\begin{verbatim}
DFA_matrix<> A;
const char *w = "word";
add_word(A, w, w + 4);
cursor<DFA_matrix<> > c(A);
for(c = A.initial(); *w && c.forward(*w); ++w);
assert (*w == 0 && c.src_final());
\end{verbatim}
\paragraph{Definition \\}
\verb+cursor.h+
\paragraph{Template parameters \\ [1ex]}
\begin{tabular}{|l|l|p{8cm}|} \hline
\bf Parameter & \bf Description & \bf Default \\ \hline
\verb+DFA+ & the automaton type & \\ \hline
\end{tabular}
\paragraph{Model of \\}
plain cursor.
\paragraph{Type requirements}
\begin{itemize}
\item \verb+DFA+ is a model of DFA
\end{itemize}
\paragraph{Public base classes \\}
\verb+plain_cursor_concept+
\paragraph{Members \\}
\begin{longtable}[h]{|l|p{1.3cm}|p{5.8cm}|} \hline
\bf Member & \bf Where defined & \bf Description \\ \hline
\endhead
% plain cursor interface specifications
% to be included into a LaTeX tabular environment with 3 columns
\verb+state_type+ & plain cursor & The type of the states handles of
the underlying DFA, that is, \verb+DFA::state_type+ \\ \hline
\verb+char_type+ & plain cursor & The type of the transitions letters,
\verb+DFA::char_type+ \\ \hline
\verb+tag_type+ & plain cursor & The type of the tags associated to states,
\verb+DFA::tag_type+ \\ \hline
\verb+char_traits+ & plain cursor & Character traits associated to
\verb+char_type+ \\ \hline
\verb+cursor()+ & plain cursor & Default constructor \\ \hline
\verb+cursor(const cursor&)+ & plain cursor & Copy constructor \\ \hline
\verb+state_type src() const+ & plain cursor & Return the state handle
the cursor is pointing to\\ \hline
\verb+cursor& operator=(state_type q)+ & plain cursor & Set the cursor
to point to state \verb+q+ \\ \hline
\verb+cursor& operator=(const cursor&)+ & plain cursor & Assignment operator \\ \hline
\verb+bool operator==(const cursor&) const+ & plain cursor & Return
\verb+true+ iff both cursors point to the same state \\ \hline
\verb+bool sink() const+ & plain cursor & Return \verb+true+ iff the
cursor points to the sink state \verb+DFA::null_state+ \\ \hline
\verb+bool forward(const char_type &a)+ & plain cursor & Move along
transition labeled with \verb+a+ if defined, otherwise move to sink
state and return \verb+false+ \\ \hline
\verb+bool exists(const char_type &a) const+ & plain cursor & Return
\verb+true+ if a transition labeled with \verb+a+ is defined \\ \hline
\verb+bool src_final() const+ & plain cursor & Return \verb+true+ if
pointed state is final \\ \hline
\verb+tag_type src_tag() const+ & plain cursor & Return the object
associated to pointed state \\ \hline

\end{longtable}
\paragraph{New members \\}
These members are not defined in the plain cursor requirements but are
specific to \verb+cursor+. \\ [1ex]
\begin{tabular}{|l|p{8.5cm}|} \hline
\bf Member & \bf Description \\ \hline
\verb+cursor(const DFA &A)+ & Construct a cursor pointing to a DFA
            {\tt A} \\ \hline
\verb+cursor(const DFA &A, state_type q)+ & Construct a cursor
pointing to the state {\tt q} of the DFA {\tt A} \\ \hline
\end{tabular}
\paragraph{Helper functions}
\begin{verbatim}
template <typename DFA>
cursor<DFA> plainc(const DFA &A, DFA::state_type q = A.initial());
\end{verbatim}
\paragraph{Notes \\}
A default-constructed cursor, or a cursor that has not been set to
point to a valid state has a singular value which means the only
operation allowed is the assignment. The sink state is not considered
as a valid state.
\paragraph{See also \\}
\verb+forward_cursor+, DFA.
%%%%%%%%%%%%%%%%%%%%%%%%%%%%%%%%%%%%%%%%%%%%%%%%%%%%%%%%%%%%%%%%%
%
%%%%%%%%%%%%%%%%%%%%%%%%%% FORWARD_CURSOR %%%%%%%%%%%%%%%%%%%%%%%%%%%%%%%%%%%%%%%%%%%%%%
\newpage
\noindent {\huge {\bf forward\_cursor$<$DFA$>$}} \\
\noindent \begin{tabular}{p{7.25cm}p{7.25cm}} \hline
\flushleft {\bf Category : {\Large cursors}} & \flushright {\bf Component Type :
{\Large Type}} 
\end{tabular} \\
\paragraph{Description \\} 
A \verb+forward_cursor+ is a pointer to an automaton transition, that
is, a triple (source state, letter, aim state). It provides all the
functionnalities of the plain cursor and some means to iterate through
the sequence of the outgoing transitions of the source state.
\paragraph{Example}
\begin{verbatim}
string w[4] = { "forward", "cursor", "code", "example" };
DFA_bin a;
add_word(a, w, w + 4);
forward_cursor c(a, a.initial());
assert(c.first());
do cout << "src " << c.src() << " letter " << c.letter
        << " aim " << c.aim();
while (c.next());
\end{verbatim}
\paragraph{Definition \\}
\verb+cursor.h+
\paragraph{Template parameters \\ [1ex]}
\begin{tabular}{|l|l|p{8cm}|} \hline
\bf Parameter & \bf Description & \bf Default \\ \hline
\verb+DFA+ & The automaton type & \\ \hline
\end{tabular}
\paragraph{Model of \\}
plain cursor, forward cursor.
\paragraph{Type requirements}
\begin{itemize}
\item \verb+DFA+ is a model of DFA.
\end{itemize}
\paragraph{Public base classes \\}
\verb+forward_cursor_concept+.
\paragraph{Members \\ [1ex]}
\begin{longtable}[h]{|l|p{1.3cm}|p{5.8cm}|} \hline
\bf Member & \bf Where defined & \bf Description \\ \hline
\endhead
% plain cursor interface specifications
% to be included into a LaTeX tabular environment with 3 columns
\verb+state_type+ & plain cursor & The type of the states handles of
the underlying DFA, that is, \verb+DFA::state_type+ \\ \hline
\verb+char_type+ & plain cursor & The type of the transitions letters,
\verb+DFA::char_type+ \\ \hline
\verb+tag_type+ & plain cursor & The type of the tags associated to states,
\verb+DFA::tag_type+ \\ \hline
\verb+char_traits+ & plain cursor & Character traits associated to
\verb+char_type+ \\ \hline
\verb+cursor()+ & plain cursor & Default constructor \\ \hline
\verb+cursor(const cursor&)+ & plain cursor & Copy constructor \\ \hline
\verb+state_type src() const+ & plain cursor & Return the state handle
the cursor is pointing to\\ \hline
\verb+cursor& operator=(state_type q)+ & plain cursor & Set the cursor
to point to state \verb+q+ \\ \hline
\verb+cursor& operator=(const cursor&)+ & plain cursor & Assignment operator \\ \hline
\verb+bool operator==(const cursor&) const+ & plain cursor & Return
\verb+true+ iff both cursors point to the same state \\ \hline
\verb+bool sink() const+ & plain cursor & Return \verb+true+ iff the
cursor points to the sink state \verb+DFA::null_state+ \\ \hline
\verb+bool forward(const char_type &a)+ & plain cursor & Move along
transition labeled with \verb+a+ if defined, otherwise move to sink
state and return \verb+false+ \\ \hline
\verb+bool exists(const char_type &a) const+ & plain cursor & Return
\verb+true+ if a transition labeled with \verb+a+ is defined \\ \hline
\verb+bool src_final() const+ & plain cursor & Return \verb+true+ if
pointed state is final \\ \hline
\verb+tag_type src_tag() const+ & plain cursor & Return the object
associated to pointed state \\ \hline

\verb+char_type letter() const+ & forward cursor & Return the letter
on the pointed transition. The cursor must have been set to point to a
defined transition beforehand by succesfully calling
\verb+first+, \verb+next()+ or \verb+find+. \\
\hline 
\verb+bool first()+ & forward cursor & Make the cursor
point to the first element of the transitions sequence of the source
state. Return \verb+true+ if there is such an element (if any
transition is defined), otherwise the pointed transition is
undefined. \\ \hline 
\verb+bool next()+ & forward cursor & Move the cursor to
the next element of the transitions sequence of source state. Return
\verb+true+ if there is such an element (the cursor is not at the end
of the sequence), otherwise the pointed transition is
undefined. \verb+first+ must have been
successfully called prior to using this method. \\ \hline
\verb+bool find(const Alphabet &a)+ & forward cursor & Make the cursor
point to the transition labelled with \verb+a+. Return \verb+true+ if
such a transition exists, otherwise the pointed transition is
undefined. \\ \hline 
\verb+void forward()+ & forward cursor & Move forward on the currently
pointed transition. The cursor must have been set to point to a
defined transition beforehand by succesfully calling
\verb+first+, \verb+next()+ or \verb+find+. \\
\hline 
\verb+bool aim_final() const+ & forward cursor & Return \verb+true+ if
the aim state is final. The cursor must have been set to point to a
defined transition beforehand by succesfully calling
\verb+first+, \verb+next()+ or \verb+find+. \\
\hline 
\verb+tag_type aim_tag() const+ & forward cursor & Return the object
associated with the aim state. The cursor must have been set to point to a
defined transition beforehand by succesfully calling
\verb+first+, \verb+next()+ or \verb+find+. \\
\hline 
\verb+state_type aim() const+ & forward cursor & Return the handle of
the aim state the cursor is point to. The cursor must have been set to point to a
defined transition beforehand by succesfully calling
\verb+first+, \verb+next()+ or \verb+find+. \\ \hline

\end{longtable}
\paragraph{New members \\}
These members are not defined in the forward cursor requirements but are
specific to \verb+forward_cursor+. \\ [1ex]
\begin{tabular}{|l|p{7cm}|} \hline
\bf Member & \bf Description \\ \hline
\verb+forward_cursor(const DFA &A)+ & Construct a forward cursor
pointing to a DFA {\tt A} \\ \hline
\verb+forward_cursor(const DFA &A, state_type q)+ & Construct a
forward cursor pointing to the state {\tt q} of the DFA {\tt A} \\
\hline 
\end{tabular}
\paragraph{Helper functions}
\begin{verbatim}
template <typename DFA>
forward_cursor<DFA> forwardc(const DFA &A, DFA::state_type q = A.initial());
\end{verbatim}
\paragraph{Notes \\}
A default-constructed forward cursor, or a forward cursor that has not
been set to point to a valid state and a valid transition has a
singular value which means the only operation allowed is the
assignment. The sink state is not considered as a valid state.
\paragraph{See also \\}
DFA, \verb+cursor+, \verb+stack_cursor+, \verb+queue_cursor+.

%%%%%%%%%%%%%%%%%%%%%%%%% STACK_CURSOR %%%%%%%%%%%%%%%%%%%%%%%%%%%%%%%%%%%%%%%%%%%%%%%
\newpage
\noindent {\huge {\bf stack\_cursor$<$ForwardCursor, Container$>$}} \\
\noindent \begin{tabular}{p{7.25cm}p{7.25cm}} \hline
\flushleft {\bf Category : {\Large cursors}} & \flushright {\bf Component Type :
{\Large Type}} 
\end{tabular} \\
\paragraph{Description \\} 
A \verb+stack_cursor+ is a forward cursor storing its path in a stack
of cursors. Each forward move along a transition pushes a new forward
cursor onto the stack top and an extra method \verb+backward+ allows
to pop. The depth-first traversal cursor \verb+dfirst_cursor+
relies on the \verb+stack_cursor+.
%\paragraph{Example}
%\begin{verbatim}
%\end{verbatim}
\paragraph{Definition \\}
\verb+cursor.h+
\paragraph{Template parameters \\ [1ex]}
\begin{tabular}{|l|p{6.5cm}|p{4.5cm}|} \hline
\bf Parameter & \bf Description & \bf Default \\ \hline
\verb+ForwardCursor+ & The type of the cursors which are stored in the
stack & \\ \hline
\verb+Container+ & The type of the sequential container implementing the stack &
\verb+vector<ForwardCursor>+ \\ \hline 
\end{tabular}
\paragraph{Model of \\}
plain cursor, forward cursor, stack cursor.
\paragraph{Type requirements \\}
\begin{itemize}
\item \verb+ForwardCursor+ is a model of forward cursor.
\item \verb+Container+ is a model of back insertion sequence.
\item \verb+Container::value_type+ must be \verb+ForwardCursor+.
\end{itemize}
\paragraph{Public base classes \\}
\verb+cursor_concept+, \verb+forward_cursor_concept+, \verb+stack_cursor_concept+.
\paragraph{Members \\ [1ex]}
\begin{longtable}[h]{|l|p{1.3cm}|p{5.8cm}|} \hline
\bf Member & \bf Where defined & \bf Description \\ \hline
\endhead
% plain cursor interface specifications
% to be included into a LaTeX tabular environment with 3 columns
\verb+state_type+ & plain cursor & The type of the states handles of
the underlying DFA, that is, \verb+DFA::state_type+ \\ \hline
\verb+char_type+ & plain cursor & The type of the transitions letters,
\verb+DFA::char_type+ \\ \hline
\verb+tag_type+ & plain cursor & The type of the tags associated to states,
\verb+DFA::tag_type+ \\ \hline
\verb+char_traits+ & plain cursor & Character traits associated to
\verb+char_type+ \\ \hline
\verb+cursor()+ & plain cursor & Default constructor \\ \hline
\verb+cursor(const cursor&)+ & plain cursor & Copy constructor \\ \hline
\verb+state_type src() const+ & plain cursor & Return the state handle
the cursor is pointing to\\ \hline
\verb+cursor& operator=(state_type q)+ & plain cursor & Set the cursor
to point to state \verb+q+ \\ \hline
\verb+cursor& operator=(const cursor&)+ & plain cursor & Assignment operator \\ \hline
\verb+bool operator==(const cursor&) const+ & plain cursor & Return
\verb+true+ iff both cursors point to the same state \\ \hline
\verb+bool sink() const+ & plain cursor & Return \verb+true+ iff the
cursor points to the sink state \verb+DFA::null_state+ \\ \hline
\verb+bool forward(const char_type &a)+ & plain cursor & Move along
transition labeled with \verb+a+ if defined, otherwise move to sink
state and return \verb+false+ \\ \hline
\verb+bool exists(const char_type &a) const+ & plain cursor & Return
\verb+true+ if a transition labeled with \verb+a+ is defined \\ \hline
\verb+bool src_final() const+ & plain cursor & Return \verb+true+ if
pointed state is final \\ \hline
\verb+tag_type src_tag() const+ & plain cursor & Return the object
associated to pointed state \\ \hline

\verb+char_type letter() const+ & forward cursor & Return the letter
on the pointed transition. The cursor must have been set to point to a
defined transition beforehand by succesfully calling
\verb+first+, \verb+next()+ or \verb+find+. \\
\hline 
\verb+bool first()+ & forward cursor & Make the cursor
point to the first element of the transitions sequence of the source
state. Return \verb+true+ if there is such an element (if any
transition is defined), otherwise the pointed transition is
undefined. \\ \hline 
\verb+bool next()+ & forward cursor & Move the cursor to
the next element of the transitions sequence of source state. Return
\verb+true+ if there is such an element (the cursor is not at the end
of the sequence), otherwise the pointed transition is
undefined. \verb+first+ must have been
successfully called prior to using this method. \\ \hline
\verb+bool find(const Alphabet &a)+ & forward cursor & Make the cursor
point to the transition labelled with \verb+a+. Return \verb+true+ if
such a transition exists, otherwise the pointed transition is
undefined. \\ \hline 
\verb+void forward()+ & forward cursor & Move forward on the currently
pointed transition. The cursor must have been set to point to a
defined transition beforehand by succesfully calling
\verb+first+, \verb+next()+ or \verb+find+. \\
\hline 
\verb+bool aim_final() const+ & forward cursor & Return \verb+true+ if
the aim state is final. The cursor must have been set to point to a
defined transition beforehand by succesfully calling
\verb+first+, \verb+next()+ or \verb+find+. \\
\hline 
\verb+tag_type aim_tag() const+ & forward cursor & Return the object
associated with the aim state. The cursor must have been set to point to a
defined transition beforehand by succesfully calling
\verb+first+, \verb+next()+ or \verb+find+. \\
\hline 
\verb+state_type aim() const+ & forward cursor & Return the handle of
the aim state the cursor is point to. The cursor must have been set to point to a
defined transition beforehand by succesfully calling
\verb+first+, \verb+next()+ or \verb+find+. \\ \hline

\verb+bool backward()+ & stack cursor & pop the stack top. Return
\verb+false+ if the resulting stack is empty. \\ \hline
\end{longtable}
\paragraph{New Members \\ [1ex]}
\begin{tabular}{|l|p{7cm}|} \hline
\bf Membre & \bf Description \\ \hline
\verb+stack_cursor(const ForwardCursor &c)+ & Construct a stack cursor
with a stack containing \verb+c+. \\ \hline
\verb+stack_cursor()+ & Construct a stack cursor
with an empty stack.\\ \hline
\end{tabular}
\paragraph{Helper Functions \\}
\begin{verbatim}
template <class ForwardCursor>
stack_cursor<ForwardCursor> stackc(const ForwardCursor &x);
\end{verbatim}
\paragraph{Notes}
\begin{itemize}
\item Calls to the \verb+stack_cursor+ methods are only valid iff the stack
is non-empty.
\item The default constructor is used by the depth-first cursor
  to implement ends of range: the empty stack serves as stop condition
  for the traversal.
\end{itemize}
\paragraph{See Also \\}
forward cursor, stack cursor, depth-first cursor, \verb+dfirst_cursor+.

%%%%%%%%%%%%%%%%%%%%%%%%% QUEUE_CURSOR %%%%%%%%%%%%%%%%%%%%%%%%%%%%%%%%%%%%%%%%%%%%%%%
\newpage
\noindent {\huge {\bf queue\_cursor$<$ForwardCusor, Container$>$}} \\
\noindent \begin{tabular}{p{7.25cm}p{7.25cm}} \hline
\flushleft {\bf Category : {\Large cursors}} & \flushright {\bf Component Type :
{\Large Type}} 
\end{tabular} \\
\paragraph{Description \\} 
A \verb+queue_cursor+ is a forward cursor storing its path in a queue
of cursors. Each move through the sequence of the outgoing transitions
of the source state (\verb+next+) enqueues a forward
cursor. An extra method \verb+dequeue+ allows to dequeue and to
implement the breadth-first traversal. 
%\paragraph{Example}
%\begin{verbatim}
%@@@Example
%\end{verbatim}
\paragraph{Definition \\}
\verb+cursor.h+
\paragraph{Template parameters \\ [1ex]}
\begin{tabular}{|l|p{7cm}|p{4cm}|} \hline
\bf Parameter & \bf Description & \bf Default \\ \hline
\verb+ForwardCursor+ & The type of the cursors stored in the queue & \\ \hline
\verb+Container+ & The type of the sequential container implementing
the queue & \verb+deque<ForwardCursor>+ \\ \hline
\end{tabular}
\paragraph{Model of \\}
plain cursor, forward cursor, queue cursor.
\paragraph{Type requirements \\}
\begin{itemize}
\item \verb+ForwardCursor+ is a model of forward cursor.
\item \verb+Container+ is a model of front insertion sequence. 
\item  \verb+Container::value_type+ must be \verb+ForwardCursor+.
\end{itemize}
\paragraph{Public base classes \\}
\verb+cursor_concept+, \verb+forward_cursor_concept+, \verb+queue_cursor_concept+.
\paragraph{Members \\ [1ex]}
\begin{longtable}[h]{|l|p{1.3cm}|p{5.8cm}|} \hline
\bf Member & \bf Where defined & \bf Description \\ \hline
\endhead
% plain cursor interface specifications
% to be included into a LaTeX tabular environment with 3 columns
\verb+state_type+ & plain cursor & The type of the states handles of
the underlying DFA, that is, \verb+DFA::state_type+ \\ \hline
\verb+char_type+ & plain cursor & The type of the transitions letters,
\verb+DFA::char_type+ \\ \hline
\verb+tag_type+ & plain cursor & The type of the tags associated to states,
\verb+DFA::tag_type+ \\ \hline
\verb+char_traits+ & plain cursor & Character traits associated to
\verb+char_type+ \\ \hline
\verb+cursor()+ & plain cursor & Default constructor \\ \hline
\verb+cursor(const cursor&)+ & plain cursor & Copy constructor \\ \hline
\verb+state_type src() const+ & plain cursor & Return the state handle
the cursor is pointing to\\ \hline
\verb+cursor& operator=(state_type q)+ & plain cursor & Set the cursor
to point to state \verb+q+ \\ \hline
\verb+cursor& operator=(const cursor&)+ & plain cursor & Assignment operator \\ \hline
\verb+bool operator==(const cursor&) const+ & plain cursor & Return
\verb+true+ iff both cursors point to the same state \\ \hline
\verb+bool sink() const+ & plain cursor & Return \verb+true+ iff the
cursor points to the sink state \verb+DFA::null_state+ \\ \hline
\verb+bool forward(const char_type &a)+ & plain cursor & Move along
transition labeled with \verb+a+ if defined, otherwise move to sink
state and return \verb+false+ \\ \hline
\verb+bool exists(const char_type &a) const+ & plain cursor & Return
\verb+true+ if a transition labeled with \verb+a+ is defined \\ \hline
\verb+bool src_final() const+ & plain cursor & Return \verb+true+ if
pointed state is final \\ \hline
\verb+tag_type src_tag() const+ & plain cursor & Return the object
associated to pointed state \\ \hline

\verb+char_type letter() const+ & forward cursor & Return the letter
on the pointed transition. The cursor must have been set to point to a
defined transition beforehand by succesfully calling
\verb+first+, \verb+next()+ or \verb+find+. \\
\hline 
\verb+bool first()+ & forward cursor & Make the cursor
point to the first element of the transitions sequence of the source
state. Return \verb+true+ if there is such an element (if any
transition is defined), otherwise the pointed transition is
undefined. \\ \hline 
\verb+bool next()+ & forward cursor & Move the cursor to
the next element of the transitions sequence of source state. Return
\verb+true+ if there is such an element (the cursor is not at the end
of the sequence), otherwise the pointed transition is
undefined. \verb+first+ must have been
successfully called prior to using this method. \\ \hline
\verb+bool find(const Alphabet &a)+ & forward cursor & Make the cursor
point to the transition labelled with \verb+a+. Return \verb+true+ if
such a transition exists, otherwise the pointed transition is
undefined. \\ \hline 
\verb+void forward()+ & forward cursor & Move forward on the currently
pointed transition. The cursor must have been set to point to a
defined transition beforehand by succesfully calling
\verb+first+, \verb+next()+ or \verb+find+. \\
\hline 
\verb+bool aim_final() const+ & forward cursor & Return \verb+true+ if
the aim state is final. The cursor must have been set to point to a
defined transition beforehand by succesfully calling
\verb+first+, \verb+next()+ or \verb+find+. \\
\hline 
\verb+tag_type aim_tag() const+ & forward cursor & Return the object
associated with the aim state. The cursor must have been set to point to a
defined transition beforehand by succesfully calling
\verb+first+, \verb+next()+ or \verb+find+. \\
\hline 
\verb+state_type aim() const+ & forward cursor & Return the handle of
the aim state the cursor is point to. The cursor must have been set to point to a
defined transition beforehand by succesfully calling
\verb+first+, \verb+next()+ or \verb+find+. \\ \hline

\verb+bool dequeue()+ & queue cursor & dequeue the current cursor and
return \verb+false+ is the resulting queue is empty \\ \hline
\end{longtable}
\paragraph{New Members \\ [1ex]}
\begin{tabular}{|l|p{7cm}|} \hline
\bf Membre & \bf Description \\ \hline
\verb+queue_cursor(const ForwardCursor &c)+ & Construct a cursor with
a queue storing \verb+c+. \\ \hline
\verb+queue_cursor()+ & Construct a cursor with an empty queue. \\ \hline
\end{tabular}
\paragraph{Helper Functions \\}
\begin{verbatim}
template <class ForwardCusor>
queue_cursor<ForwardCusor> queuec(const ForwardCusor &x);
\end{verbatim}
\paragraph{Notes}
\begin{itemize}
\item Calls to the \verb+queue_cursor+ methods are only valid iff the queue
is non-empty.
\item The default constructor is used by the breadth-first cursor
  to implement ends of range: the empty queue serves as stop condition
  for the traversal.
\end{itemize}
\paragraph{See Also \\}
forward cursor, queue cursor, breadth-first cursor, \verb+bfirst_cursor+.

%%%%%%%%%%%%%%%%%%%%%%%%% DFIRST_CURSOR %%%%%%%%%%%%%%%%%%%%%%%%%%%%%%%%%%%%%%%%%%%%%%%
\newpage
\noindent {\huge {\bf dfirst\_cursor$<$StackCursor$>$}} \\
\noindent \begin{tabular}{p{7.25cm}p{7.25cm}} \hline
\flushleft {\bf Category : {\Large cursors}} & \flushright {\bf Component Type :
{\Large Type}} 
\end{tabular} \\
\paragraph{Description \\} 
A \verb+dfirst_cursor+ implements the depth-first traversal on acyclic
deterministic automata. It is in
some sense an iterator on a sequence of transitions ordered according
to the depth-first traversal algorithm. The method \verb+forward+
allows to increment the cursor, making it point to the next transition
in the sequence. This methods returns \verb+true+ if the transition
reached has been pushed onto the stack (forward move) and \verb+false+
otherwise (pop and backward move). \\
The \verb+dfirst_cursor+ is fundamental because it is used much in the
same way as the iterators on sequence to define ranges for
algorithms. 
\paragraph{Example}
\begin{verbatim}
string w[4] = { "forward", "cursor", "code", "example" };
DFA_bin a;
add_word(a, w, w + 4);
dfirst_cursor<stack_cursor<forward_cursor<DFA_bin<> > > >
  first = dfirstc(a), last;
vector<char> word;
while (first != last) {
  do {
    word.push_back(first.letter());
    if (first.aim_final()) {
      copy(word.begin(), word.end(), ostream_iterator<char>(cout));
      cout << endl;
    }
  } while (first.forward());

  while (!first.forward()) word.pop_back();
}
\end{verbatim}
\paragraph{Definition \\}
\verb+cursor.h+
\paragraph{Template parameters \\ [1ex]}
\begin{tabular}{|l|l|l|} \hline
\bf Parameter & \bf Description & \bf Default \\ \hline
\verb+StackCursor+ & The type of the underlying stack cursor & \\
\hline 
\end{tabular}
\paragraph{Model of \\}
depth-first cursor.
\paragraph{Type requirements \\}
\begin{itemize}
\item \verb+StackCursor+ is a model of stack cursor.
\end{itemize}
\paragraph{Public base classes \\}
\verb+dfirst_cursor_concept+.
\paragraph{Members \\ [1ex]}
\begin{longtable}{|p{5.5cm}|p{1.7cm}|p{6.5cm}|} \hline
\bf Member & \bf Where defined & \bf Description \\ \hline
\endhead
\verb+state_type+ & depth-first cursor & The type of the states
handles of the underlying DFA. \\ \hline
\verb+char_type+ & depth-first cursor & The type of the transitions
letters. \\ \hline 
\verb+tag_type+ & depth-first cursor & The type of the tags associated to
states. \\ \hline 
\verb+dfirst_cursor()+ & depth-first cursor & Construct a cursor with an
empty stack useful as end-of-range iterator. \\ \hline
\verb+dfirst_cursor(+ \verb+      const dfirst_cursor &c)+ &  depth-first cursor &
Copy constructor. \\ \hline 
\verb+state_type src() const+ & depth-first cursor & Return the state
handle the cursor is pointing to. \\ \hline
\verb+bool src_final() const+ & depth-first cursor & Return
\verb+true+ if the pointed state is final. \\ \hline
\verb+char_type letter() const+ & depth-first cursor & Return the
letter on the pointed transition. \\ \hline
\verb+bool aim() const+ & depth-first cursor & Return the handle of
the aim state the cursor is pointing to. \\ \hline
\verb+bool aim_final() const+  & depth-first cursor & Return
\verb+true+ if the aim state is final. \\ \hline
\verb+bool operator==(+ 
\verb+ const dfirst_cursor &c) const+ & depth-first cursor & Return
\verb+true+ iff both stacks are equal. \\ \hline
\verb+bool forward()+ & depth-first cursor & Increment the cursor
making it point to the next transition in the sequence. Return
\verb+true+ if the transition reached has been pushed onto the
stack. \\ \hline 
\verb+Tag src_tag() const+ & depth-first cursor & Return the tag
associated to the source state. \\ \hline
\verb+Tag aim_tag() const+ & depth-first cursor & Return the tag
associated to the aim state. \\ \hline

\end{longtable}
\paragraph{New Members \\ [1ex]}
\begin{tabular}{|l|p{7cm}|} \hline
\bf Membre & \bf Description \\ \hline
\verb+dfirst_cursor(const StackCursor &c)+ & Construct a cursor with
\verb+c+ as stack. \\ \hline
\end{tabular}
\paragraph{Helper Functions \\}
The function \verb+dfirstc+ returns a \verb+dfirst_cursor+ constructed
from the object passed as argument. This object is allowed to be a
model of DFA, forward cursor or stack cursor.
\paragraph{Notes}
\begin{itemize}
\item Calls to method other than \verb+operator==+ are valid iff the cursor
stack is non-empty.
\item This cursor works only on acyclic structures, use
  \verb+dfirst_mark_cursor+ on cyclic DFAs.
\end{itemize}
\paragraph{See Also \\}
depth-first cursor, stack cursor, \verb+dfirst_mark_cursor+.

%%%%%%%%%%%%%%%%%%%%%%%%% DFIRST_MARK_CURSOR %%%%%%%%%%%%%%%%%%%%%%%%%%%%%%%%%%%%%%%%%%%%%%%
\newpage
\noindent {\huge {\bf dfirst\_mark\_cursor$<$StackCursor, Marker$>$}} \\
\noindent \begin{tabular}{p{7.25cm}p{7.25cm}} \hline
\flushleft {\bf Category : {\Large cursors}} & \flushright {\bf Component Type :
{\Large Type}} 
\end{tabular} \\
\paragraph{Description \\} 
A \verb+dfirst_mark_cursor+ implements the depth-first traversal on cyclic
deterministic automata. It works exactly as the \verb+dfirst_cursor+
does except that a state-mark function garantees that each transition
is only reached twice
%\paragraph{Example}
%\begin{verbatim}
%@@@Example
%\end{verbatim}
\paragraph{Definition \\}
\verb+cursor.h+
\paragraph{Template parameters \\ [1ex]}
\begin{tabular}{|l|l|p{5cm}|} \hline
\bf Parameter & \bf Description & \bf Default \\ \hline
\verb+StackCursor+ & The type of the underlying stack cursor & \\ \hline
\verb+Marker+ & The type of the state-mark object function & \verb+set_marker+ \\ \hline
\end{tabular}
\paragraph{Model of \\}
depth-first cursor.
\paragraph{Type requirements \\}
\begin{itemize}
\item \verb+StackCursor+ is a model of stack cursor.
\item \verb+Marker+ is a model of state marker.
\end{itemize}
\paragraph{Public base classes \\}
\verb+dfirst_cursor_concept+.
\paragraph{Members \\}
\begin{longtable}{|p{5.5cm}|p{1.7cm}|p{6.5cm}|} \hline
\bf Member & \bf Where defined & \bf Description \\ \hline
\endhead
\verb+state_type+ & depth-first cursor & The type of the states
handles of the underlying DFA. \\ \hline
\verb+char_type+ & depth-first cursor & The type of the transitions
letters. \\ \hline 
\verb+tag_type+ & depth-first cursor & The type of the tags associated to
states. \\ \hline 
\verb+dfirst_cursor()+ & depth-first cursor & Construct a cursor with an
empty stack useful as end-of-range iterator. \\ \hline
\verb+dfirst_cursor(+ \verb+      const dfirst_cursor &c)+ &  depth-first cursor &
Copy constructor. \\ \hline 
\verb+state_type src() const+ & depth-first cursor & Return the state
handle the cursor is pointing to. \\ \hline
\verb+bool src_final() const+ & depth-first cursor & Return
\verb+true+ if the pointed state is final. \\ \hline
\verb+char_type letter() const+ & depth-first cursor & Return the
letter on the pointed transition. \\ \hline
\verb+bool aim() const+ & depth-first cursor & Return the handle of
the aim state the cursor is pointing to. \\ \hline
\verb+bool aim_final() const+  & depth-first cursor & Return
\verb+true+ if the aim state is final. \\ \hline
\verb+bool operator==(+ 
\verb+ const dfirst_cursor &c) const+ & depth-first cursor & Return
\verb+true+ iff both stacks are equal. \\ \hline
\verb+bool forward()+ & depth-first cursor & Increment the cursor
making it point to the next transition in the sequence. Return
\verb+true+ if the transition reached has been pushed onto the
stack. \\ \hline 
\verb+Tag src_tag() const+ & depth-first cursor & Return the tag
associated to the source state. \\ \hline
\verb+Tag aim_tag() const+ & depth-first cursor & Return the tag
associated to the aim state. \\ \hline

\end{longtable}
\paragraph{New Members \\ [1ex]}
\begin{tabular}{|l|p{7cm}|} \hline
\bf Membre & \bf Description \\ \hline
\verb+dfirst_cursor(const StackCursor &c)+ & Construct a cursor with
\verb+c+ as stack. \\ \hline
\end{tabular}
\paragraph{Helper Functions \\}
The function \verb+dfirst_markc+ returns a \verb+dfirst_mark_cursor+ constructed
from the object passed as argument. This object is allowed to be a
model of DFA, forward cursor or stack cursor.
\paragraph{Notes}
\begin{itemize}
\item Calls to method other than \verb+operator==+ are valid iff the cursor
stack is non-empty.
\item This cursor is designed for cyclic structures. Using it on
  acyclic DFAs results in memory and time penalties but should not
  disrupt the processing beyond that. On a acyclic structures, a
  \verb+dfirst_cursor+ is more appropriated.
\end{itemize}
\paragraph{See Also \\}
depth-first cursor, stack cursor, \verb+dfirst_cursor+.

%%%%%%%%%%%%%%%%%%%%%%%%%% BFIRST_CURSOR %%%%%%%%%%%%%%%%%%%%%%%%%%%%%%%%%%%%%%%%%%%%%%
\newpage
\noindent {\huge {\bf bfirst\_cursor$<$QueueCursor$>$}} \\
\noindent \begin{tabular}{p{7.25cm}p{7.25cm}} \hline
\flushleft {\bf Category : {\Large cursors}} & \flushright {\bf Component Type :
{\Large Type}} 
\end{tabular} \\
\paragraph{Description \\} 
A \verb+bfirst_cursor+ implements the breadth-first traversal on acyclic
deterministic automata. It is an iterator on a sequence of transitions
ordered according to the breadth-first traversal algorithm. The method
\verb+next+ 
allows to increment the cursor, making it point to the next transition
in the sequence. This methods returns \verb+true+ if the transition
reached has been enqueued and \verb+false+
otherwise (dequeue). \\
The \verb+bfirst_cursor+ is used in the same way as the iterators on
sequence to define ranges for algorithms. 
\paragraph{Example}
\begin{verbatim}
string w[4] = { "forward", "cursor", "code", "example" };
DFA_bin a;
tree_build(a, w, w + 4);
bfirst_cursor<queue_cursor<forward_cursor<DFA_bin<> > > >
  first = bfirstc(a), last;
while (first != last) {
  do 
    cout << "src " << first.src() << " letter " << first.letter()
         << " aim " << first.aim() << endl;
  while (first.next());
}
\end{verbatim}
\paragraph{Definition \\}
\verb+cursor.h+
\paragraph{Template parameters \\ [1ex]}
\begin{tabular}{|l|l|p{5cm}|} \hline
\bf Parameter & \bf Description & \bf Default \\ \hline
\verb+QueueCursor+ & the type of the underlying queue cursor & \\ \hline
\end{tabular}
\paragraph{Model of \\}
breadth-first cursor.
\paragraph{Type requirements}
\begin{itemize}
\item \verb+QueueCursor+ is a model of queue cursor.
\end{itemize}
\paragraph{Public base classes \\}
\verb+bfirst_cursor_concept+.
\paragraph{Members \\ [1ex]}
\begin{longtable}{|p{5.5cm}|p{1.7cm}|p{6.5cm}|} \hline
\bf Member & \bf Where defined & \bf Description \\ \hline
\endhead
\verb+state_type+ & breadth-first cursor & The type of the states
handles of the underlying DFA. \\ \hline
\verb+char_type+ & breadth-first cursor & The type of the transitions
letters. \\ \hline 
\verb+tag_type+ & breadth-first cursor & The type of the tags associated to
states. \\ \hline 
\verb+bfirst_cursor()+ & breadth-first cursor & Construct a cursor with an
empty queue useful as end-of-range iterator. \\ \hline
\verb+bfirst_cursor(+ \verb+      const bfirst_cursor &c)+ &  breadth-first cursor &
Copy constructor. \\ \hline 
\verb+state_type src() const+ & breadth-first cursor & Return the state
handle the cursor is pointing to. \\ \hline
\verb+bool src_final() const+ & breadth-first cursor & Return
\verb+true+ if the pointed state is final. \\ \hline
\verb+char_type letter() const+ & breadth-first cursor & Return the
letter on the pointed transition. \\ \hline
\verb+bool aim() const+ & breadth-first cursor & Return the handle of
the aim state the cursor is pointing to. \\ \hline
\verb+bool aim_final() const+  & breadth-first cursor & Return
\verb+true+ if the aim state is final. \\ \hline
\verb+bool operator==(+ 
\verb+ const bfirst_cursor &c) const+ & breadth-first cursor & Return
\verb+true+ iff both queues are equal. \\ \hline
\verb+Tag src_tag() const+ & breadth-first cursor & Return the tag
associated to the source state. \\ \hline
\verb+Tag aim_tag() const+ & breadth-first cursor & Return the tag
associated to the aim state. \\ \hline
\verb+bool next_transition()+ & breadth-first cursor & Increment the cursor
making it point to the next transition in the sequence. Return
\verb+true+ if there are transitions left. \\ \hline

\end{longtable}
\paragraph{New Members \\ [1ex]}
\begin{tabular}{|l|p{7cm}|} \hline
\bf Member & \bf Description \\ \hline
\verb+bfirst_cursor(const QueueCursor &c)+ & Construct a cursor with
\verb+c+ as queue. \\ \hline
\end{tabular}
\paragraph{Helper Functions \\}
The function \verb+bfirst_c+ returns a \verb+bfirst_cursor+ constructed
from the object passed as argument. This object is allowed to be a
model of DFA, forward cursor or queue cursor.
\paragraph{Notes}
\begin{itemize}
\item Calls to method other than \verb+operator==+ are valid iff the cursor
queue is non-empty.
\item This cursor works only on acyclic structures, use
  \verb+bfirst_mark_cursor+ on cyclic DFAs.
\end{itemize}
\paragraph{See Also \\}
breadth-first cursor, queue cursor, \verb+bfirst_mark_cursor+.
\subsection{Cursor Adapters}
\subsection{Algorithms}
\include{algorithms}
%%%%%%%%%%%%%%%%%%%%%%%%%%%% CCOPY %%%%%%%%%%%%%%%%%%%%%%%%%%%%%%%%%%%%%%%%%%%%
% \newpage
% \noindent {\huge {\bf ccopy}} \\
% \noindent \begin{tabular}{p{7.25cm}p{7.25cm}} \hline
% \flushleft {\bf Category : {\Large Algorithm}} & \flushright {\bf Component Type :
% {\Large Function}} 
% \end{tabular} \\
% \paragraph{Prototype}
% \begin{verbatim}
% template <class DFirstCursor, class DFA>
% DFA::state_type 
% ccopy(DFA &out, DFirstCursor first, DFirstCursor last = DFirstCursor());
% \end{verbatim}
% \paragraph{Description \\}
% Copies the source automaton defined by the range \verb+[first, last)+
% into the destination automaton \verb+out+ by performing a depth-first
% traversal during which only the paths leading to final states are
% copied : this algorithm trims uneeded states and transitions. On the
% other hand, it only works on acyclic DFAs. See \verb+clone+ for cyclic
% structures copy. \\
% \verb+ccopy+ returns the first state of the iteration (the last
% duplicated state).
% \paragraph{Definition \\}
% \verb+ccopy.h+
% \paragraph{Requirements on types}
% \begin{itemize}
% \item \verb+DFirstCursor+ is a model of depth-first cursor.
% \item \verb+DFA+ is a model of DFA.
% \item \verb+DFirstCursor::char_type+ is convertible to
%   \verb+DFA::char_type+. 
% \end{itemize}
% \paragraph{Preconditions}
% \begin{itemize}
% \item \verb+[first, last)+ is a valid range.
% \item The source DFA is acyclic.
% \end{itemize}
% \paragraph{Complexity \\}
% Linear: at most \verb+last - first+ transitions are duplicated.
% \paragraph{Example}
% \begin{verbatim}
% #include <astl.h>
% #include <dfa.h>
% #include <cursor.h>
% #include <ccopy.h>
% #include <set_operation.h>
% #include <stream.h>
% #include <iostream>

% int main()
% {
%   DFA_map<> S;
%   DFA_bin<> D1, D2;

%   DFA_bin<>::state_type i = ccopy(D1, dfirstc(S));
%   D1.initial(i);

%   i = ccopy(D2, clonec(std::cin));  // read from stdin
%   D2.initial(i);

%   i = ccopy(D3, dfirstc(notc(S)));  // negation
%   D3.initial(i);

%   DFA_stream out(std::cout);
%   ccopy(out, dfirstc(S));  // write to stdout
% }
% \end{verbatim}
% \paragraph{Notes}
% \begin{itemize}
% \item This algorithm does not care about the initial state, it is up to the
% user to manage any operations related to it.
% \item This algorithm is not just a duplication procedure, it is the 
% only way (along side \verb+clone+ which is a slight variant of
% \verb+ccopy+) to apply an algorithm and to get the result of it as an
% automaton container. It is also enough to use almost all of the
% operations implemented in ASTL.
% \end{itemize}
% \paragraph{See Also \\}
% \verb+clone+
% %%%%%%%%%%%%%%%%%%%%%%%%%%%% CLONE %%%%%%%%%%%%%%%%%%%%%%%%%%%%%%%%%%%%%%%%%%%%
% \newpage
% \noindent {\huge {\bf clone}} \\
% \noindent \begin{tabular}{p{7.25cm}p{7.25cm}} \hline
% \flushleft {\bf Category : {\Large Algorithm}} & \flushright {\bf Component Type :
% {\Large Function}} 
% \end{tabular} \\
% \paragraph{Prototype}
% \begin{verbatim}
% template <class DFirstCursor, class DFA>
% DFA::state_type 
% clone(DFA &out, DFirstCursor first, DFirstCursor last = DFirstCursor());
% \end{verbatim}
% \paragraph{Description \\}
% Copies the source automaton defined by the range \verb+[first, last)+
% into the destination automaton \verb+out+ by performing a depth-first
% traversal during which all transitions are duplicated.
% Returns the first state of the iteration (the first duplicated state).
% \paragraph{Definition \\}
% \verb+ccopy.h+
% \paragraph{Requirements on types}
% \begin{itemize}
% \item \verb+DFirstCursor+ is a model of depth-first cursor.
% \item \verb+DFA+ is a model of DFA.
% \item \verb+DFirstCursor::char_type+ is convertible to
%   \verb+DFA::char_type+. 
% \end{itemize}
% \paragraph{Preconditions}
% \begin{itemize}
% \item \verb+[first, last)+ is a valid range.
% \item Either the source automaton is acyclic or \verb+DFirstCursor+ is
%   a depth-first cursor with marker.
% \end{itemize}
% \paragraph{Complexity \\}
% Linear: exactly \verb+last - first+ transitions are duplicated.
% \paragraph{Example}
% \begin{verbatim}
% #include <astl.h>
% #include <dfa.h>
% #include <cursor.h>
% #include <ccopy.h>
% #include <set_operation.h>
% #include <stream.h>
% #include <iostream>

% int main()
% {
%   DFA_map<> S;
%   DFA_bin<> D1, D2;

%   DFA_bin<>::state_type i = clone(D1, dfirstc(S));
%   D1.initial(i);

%   i = clone(D2, clonec(std::cin));  // read from stdin
%   D2.initial(i);

%   i = clone(D3, dfirst_markc(notc(S)));  // negation is cyclic
%   D3.initial(i);

%   DFA_stream out(std::cout);
%   clone(out, dfirstc(S));  // write to stdout
% }
% \end{verbatim}
% \paragraph{Notes}
% \begin{itemize}
% \item This algorithm does not care about the initial state, it is up to the
% user to manage any operations related to it.
% \item This algorithm is not just a duplication procedure, it is the 
% only way (along side \verb+ccopy+ which is a slight variant of
% \verb+clone+) to apply an algorithm and to get the result of it as an
% automaton container. It is also enough to use almost all of the
% operations implemented in ASTL.
% \end{itemize}
% \paragraph{See Also \\}
% \verb+ccopy+
% %%%%%%%%%%%%%%%%%%%%%%%%%%%%% DOT %%%%%%%%%%%%%%%%%%%%%%%%%%%%%%%%%%%%%%%%%%%
% \newpage
% \noindent {\huge {\bf dot}} \\
% \noindent \begin{tabular}{p{7.25cm}p{7.25cm}} \hline
% \flushleft {\bf Category : {\Large Algorithm}} & \flushright {\bf Component Type :
% {\Large Function}} 
% \end{tabular} \\
% \paragraph{Prototype}
% \begin{verbatim}
% template <typename DFirstCursor>
% void dot(ostream &out, DFirstCursor x, DFirstCursor y = DFirstCursor());

% template <typename DFirstCursor>
% void dot(DFA_dot &out, DFirstCursor x, DFirstCursor y = DFirstCursor());
% \end{verbatim}
% \paragraph{Description \\} 
% Output the automaton defined by the range \verb+[x, y)+ to the output
% stream \verb+out+ in a representation suitable as input to the command
% \verb+dot+ from GraphViz. \\
% The second version of \verb+dot+ allows graphical customization
% through the \verb+DFA_dot+ object passed as first argument.
% \paragraph{Definition \\}
% \verb+dot.h+
% \paragraph{Requirements on types}
% \begin{itemize}
% \item \verb+DFirstCursor+ is a model of depth-first cursor.
% \end{itemize}
% \paragraph{Preconditions}
% \begin{itemize}
% \item \verb+[x, y)+ is a valid range.
% \end{itemize}
% \paragraph{Complexity \\}
% Linear
% \paragraph{Example}
% \begin{verbatim}
% #include <astl.h>
% #include <dfa.h>
% #include <cursor.h>
% #include <dot.h>
% #include <iostream>

% int main()
% {
%   DFA_matrix<> A, B;
%   dot(std::cout, dfirstc(A));

%   DFA_dot out(std::cout);
%   out.state_fontsize(18);     // customize display
%   dot(out, dfirst_markc(B));  // if B is cyclic
% }
% \end{verbatim}
% \paragraph{Notes \\}
% \verb+dot+ is written as a call to the algorithm \verb+clone+.
% \paragraph{See Also \\}
% \verb+clone+
% %%%%%%%%%%%%%%%%%%%%%%%%%%% ACYCLIC MINIMIZATION %%%%%%%%%%%%%%%%%%%%%%%%%%%%%%%%%%%%%%%%%%%%%
% \newpage
% \noindent {\huge {\bf acyclic\_minimization}} \\
% \noindent \begin{tabular}{p{7.25cm}p{7.25cm}} \hline
% \flushleft {\bf Category : {\Large Algorithm}} & \flushright {\bf Component Type :
% {\Large Function}} 
% \end{tabular} \\
% \paragraph{Prototype}
% \begin{verbatim}
% template <typename DFA>
% void acyclic_minimization(DFA &a);
% \end{verbatim}
% \paragraph{Description \\}
% Minimizes the size of \verb+a+ by removing redundant states and
% transitions. From the languages point of a view, the minimized version
% of \verb+a+ is strictly equivalent to the original one.
% \paragraph{Definition \\}
% \verb+minimize.h+
% \paragraph{Requirements on types}
% \begin{itemize}
% \item \verb+DFA+ is a model of DFA.
% \item \verb+DFA::tag_type+ is at least \verb+minimization_tag+.
% \item The transitions of \verb+DFA+ are sorted, which
%   means that this algorithm can not be used on containers
%   \verb+DFA_mtf+, \verb+DFA_hash+ and \verb+DFA_tr+.
% \end{itemize}
% \paragraph{Preconditions}
% \begin{itemize}
% \item \verb+a+ is acyclic.
% \end{itemize}
% \paragraph{Complexity \\}
% $n\log(n)$ where $n$ is \verb+a.trans_count()+.
% \paragraph{Example}
% \begin{verbatim}
% #include <astl.h>
% #include <minimize.h>
% #include <language.h>
% #include <iostream>
% #include <cassert>

% int main()
% {
%   DFA_matrix<plain, minimization_tag> A;
%   unsigned long Q = A.state_count(), T = A.trans_count();
%   language(std::cout, dfirstc(A));
%   acyclic_minimization(A);
%   language(std::cout, dfirstc(A));  // should be the same as above
%   assert(A.state_count() <= Q && A.trans_count() <= T);
% }
% \end{verbatim}
% %\paragraph{Notes}
% %\begin{itemize}
% %\item 
% %\end{itemize}
% \paragraph{See Also \\}
% \verb+brzozowski+.

% %%%%%%%%%%%%%%%%%%%%%%%%%%%%% BRZOZOWSKI %%%%%%%%%%%%%%%%%%%%%%%%%%%%%%%%%%%%%%%%%%%
% \newpage
% \noindent {\huge {\bf brzozowski}} \\
% \noindent \begin{tabular}{p{7.25cm}p{7.25cm}} \hline
% \flushleft {\bf Category : {\Large Algorithm}} & \flushright {\bf Component Type :
% {\Large Function}} 
% \end{tabular} \\
% \paragraph{Prototype}
% \begin{verbatim}
% template <typename DFA1, typename DFA2>
% void brzozowski(const DFA1 &A, DFA2 &B);
% \end{verbatim}
% \paragraph{Description \\}
% \verb+brzozowski+ performs a minimization of the possibly cyclic DFA
% \verb+A+ and copies the result into the DFA \verb+B+. This is a
% generic implementation, on acyclic structures the algorithm
% \verb+acyclic_minimization+ is much more efficient.
% \paragraph{Definition \\}
% \verb+minimize.h+
% \paragraph{Requirements on types}
% \begin{itemize}
% \item \verb+DFA1+ and \verb+DFA2+ are models of DFA.
% \item \verb+DFA1::char_type+ is convertible to
%   \verb+DFA2::char_type+. 
% \end{itemize}
% \paragraph{Preconditions}
% \begin{itemize}
% \item \verb+B+ must be empty.
% \item \verb+A+ must have sorted transitions.
% \end{itemize}
% \paragraph{Complexity \\}
% \paragraph{Example}
% \begin{verbatim}
% #include <astl.h>
% #include <cursor.h>
% #include <minimize.h>
% #include <dfa.h>
% #include <set_operation.h>
% #include <cassert>
% #include <iostream>

% int main()
% {
%   DFA_map<> A;
%   DFA_bin<> B;
%   DFA_matrix<> C;
%   brzozowski(A, B);
%   assert(B.state_count() <= A.state_count() 
%          && B.trans_count() <= A.trans_count());
%   brzozowski(B, C);
%   assert(isomorph(dfirst_markc(B), dfirst_mark(C)));
% }
% \end{verbatim}
% \paragraph{Notes \\}
% The \verb+A+ must stores the transitions in sorted containers which
%   means that this algorithm can not be used with source containers of
%   type \verb+DFA_mtf+, \verb+DFA_hash+ or \verb+DFA_tr+.
% \paragraph{See Also \\}
% \verb+acyclic_minimization+.
% %%%%%%%%%%%%%%%%%%%%%%%%%%%%% REVERSE %%%%%%%%%%%%%%%%%%%%%%%%%%%%%%%%%%%%%%%%%%%
% \newpage
% \noindent {\huge {\bf reverse}} \\
% \noindent \begin{tabular}{p{7.25cm}p{7.25cm}} \hline
% \flushleft {\bf Category : {\Large Algorithm}} & \flushright {\bf Component Type :
% {\Large Function}} 
% \end{tabular} \\
% \paragraph{Prototype}
% \begin{verbatim}
% template <typename DFA, typename NFA>
% void reverse(const DFA &A, NFA &B);
% \end{verbatim}
% \paragraph{Description \\}
% \verb+reverse+ computes a non-deterministic automaton \verb+B+
% recognizing the mirror-image language of a DFA \verb+A+, that is : 
% $$ w = a_{1} a_{2} ...a_{n} \in L(A) \Leftrightarrow w' =
% a_{n} a_{n-1} ...a_{1} \in L(B)$$ This algorithm is reused in the
% Brzozowski's minimization algorithm. 
% \paragraph{Definition \\}
% \verb+minimize.h+
% \paragraph{Requirements on types}
% \begin{itemize}
% \item \verb+DFA+ is a model of DFA.
% \item \verb+NFA+ is a model of NFA.
% \item \verb+DFA::char_type+ is convertible to \verb+NFA::char_type+. 
% \end{itemize}
% \paragraph{Preconditions}
% \begin{itemize}
% \item \verb+B+ must be empty.
% \end{itemize}
% \paragraph{Complexity \\}
% $n\log(m)$ where $n$ is \verb+A.trans_count()+ and $m$ is
% \verb+A.state_count()+. 
% \paragraph{Example}
% \begin{verbatim}
% #include <astl.h>
% #include <ccopy.h>
% #include <minimize.h>
% #include <determinize.h>
% #include <dfa.h>
% #include <nfa.h>

% int main()
% {
%   DFA_map<> A;
%   NFA_mmap<> B;
%   DFA_bin<> C;

%   reverse(A, B);
%   clone(C, dfirstc(forwarddc(B)));  // determinization
% }
% \end{verbatim}
% %%%%%%%%%%%%%%%%%%%%%%%%% MAKE_HASH %%%%%%%%%%%%%%%%%%%%%%%%%%%%%%%%%%%%%%%%%%%%%%%
% \newpage
% \noindent {\huge {\bf make\_hash}} \\
% \noindent \begin{tabular}{p{7.25cm}p{7.25cm}} \hline
% \flushleft {\bf Category : {\Large Algorithm}} & \flushright {\bf Component Type :
% {\Large Function}} 
% \end{tabular} \\
% \paragraph{Prototype}
% \begin{verbatim}
% template <typename DFA>
% void make_hash(DFA& a);
% \end{verbatim}
% \paragraph{Description \\}
% \verb+make_hash+ computes the needed data to turn the DFA \verb+a+
% into a hashing 
% automaton. A hashing automaton provides a highly efficient way to
% implement a bidirectionnal mapping (known as perfect hash function) 
% between strings of any character type and the positive integers. Such
% an automaton can be used with the algorithms \verb+hash_value+ and
% \verb+value_hash+.
% \paragraph{Definition \\}
% \verb+hash.h+
% \paragraph{Requirements on types}
% \begin{itemize}
% \item \verb+DFA+ is a model of DFA.
% \item \verb+DFA::tag_type+ is \verb+hash_tag+ defined in \verb+hash.h+
% \item The transitions of \verb+DFA+ are sorted.
% \end{itemize}
% \paragraph{Preconditions}
% \begin{itemize}
% \item \verb+a+ is acyclic.
% \end{itemize}
% \paragraph{Complexity \\}
% $O(m)$ where $m$ is \verb+A.state_count()+.
% \paragraph{Example}
% \begin{verbatim}
% #include <astl.h>
% #include <dfa.h>
% #include <cursor.h>
% #include <hash.h>
% #include <iostream>
% #include <cassert>
% #include <string>

% int main()
% {
%   DFA_matrix<plain, hash_tag> A;
%   make_hash(A);
%   const char *word = ``word'';
%   std::cout << hash_value(forwardc(A), word, word + 4) << std::endl;
%   std::string s;
%   value_hash(forwardc(A), 14, std::back_inserter(s));
% } 
% \end{verbatim}
% \paragraph{Notes}
% \begin{itemize}
% \item The requirement that \verb+DFA::tag_type+ be
%   \verb+hash_tag+ is a minimal assumption. The tag type can be
%   {\em anything} provided that it is {\em at least} a \verb+hash_tag+.
% \item The DFA must stores the transitions in sorted containers which
%   means that this algorithm can not be used on containers
%   \verb+DFA_mtf+, \verb+DFA_hash+ and \verb+DFA_tr+.
% \item A DFA must be rehashed each time it undergoes any
%   modifications. 
% \end{itemize}
% \paragraph{See Also \\}
% \verb+hash_cursor+, \verb+hash_value+, \verb+value_hash+.
% %%%%%%%%%%%%%%%%%%%%%%%%%%%%%% HASH_VALUE %%%%%%%%%%%%%%%%%%%%%%%%%%%%%%%%%%%%%%%%%%
% \newpage
% \noindent {\huge {\bf hash\_value}} \\
% \noindent \begin{tabular}{p{7.25cm}p{7.25cm}} \hline
% \flushleft {\bf Category : {\Large Algorithm}} & \flushright {\bf Component Type :
% {\Large Function}} 
% \end{tabular} \\
% \paragraph{Prototype}
% \begin{verbatim}
% template <typename ForwardCursor, typename InputIterator>
% int hash_value(ForwardCursor c, InputIterator first, InputIterator last);
% \end{verbatim}
% \paragraph{Description \\}
% Returns the positive integer hash value associated to the sequence of
% characters \verb+[first, last)+ by the hashing automaton pointed to by
% \verb+c+. A null value is returned if the word is not recognized by
% the automaton. 
% \paragraph{Definition \\}
% \verb+hash.h+
% \paragraph{Requirements on types}
% \begin{itemize}
% \item \verb+ForwardCursor+ is a model of forward cursor.
% \item \verb+InputIterator+ is a model of input iterator.
% \item \verb+InputIterator::value_type+ is convertible to
%   \verb+ForwardCursor::char_type+.
% \end{itemize}
% \paragraph{Preconditions}
% \begin{itemize}
% \item \verb+[first, last)+ is a valid range.
% \item \verb+c+ points to a hashing automaton constructed with the
%   algorithm \verb+make_hash+.
% \item \verb+c+ points to the initial state of the hashing automaton. 
% \end{itemize}
% \paragraph{Complexity \\}
% \verb+last - first+ calls to \verb+ForwardCursor::forward+.
% \paragraph{Example}
% \begin{verbatim}
% #include <astl.h>
% #include <dfa.h>
% #include <cursor.h>
% #include <hash.h>
% #include <iostream>
% #include <cassert>

% int main()
% {
%   DFA_matrix<plain, hash_tag> A;
%   make_hash(A);
%   const char *word = ``word'';
%   std::cout << hash_value(forwardc(A), word, word + 4) << std::endl;
% } 
% \end{verbatim}
% \paragraph{See Also \\}
% \verb+make_hash+, \verb+value_hash+
% %%%%%%%%%%%%%%%%%%%%%%%%%%% VALUE_HASH %%%%%%%%%%%%%%%%%%%%%%%%%%%%%%%%%%%%%%%%%%%%%
% \newpage
% \noindent {\huge {\bf value\_hash}} \\
% \noindent \begin{tabular}{p{7.25cm}p{7.25cm}} \hline
% \flushleft {\bf Category : {\Large Algorithm}} & \flushright {\bf Component Type :
% {\Large Function}} 
% \end{tabular} \\
% \paragraph{Prototype}
% \begin{verbatim}
% template <typename ForwardCursor, typename OutputIterator>
% int value_hash(ForwardCursor c, int i, OutputIterator x);
% \end{verbatim}
% \paragraph{Description \\}
% Returns the word associated to the positive integer \verb+i+ by the
% hashing automaton pointed to by \verb+c+. The returned word is copied
% to the output sequence pointed by \verb+x+ and the function returned
% value is its length. 
% \paragraph{Definition \\}
% \verb+hash.h+
% \paragraph{Requirements on types}
% \begin{itemize}
% \item \verb+ForwardCursor+ is a model of forward cursor.
% \item \verb+OutputIterator+ is a model of output iterator.
% \item \verb+ForwardCursor::char_type+ is convertible to
%   \verb+OutputIterator::value_type+.
% \end{itemize}
% \paragraph{Preconditions}
% \begin{itemize}
% \item \verb+c+ points to a hashing automaton constructed with the
%   algorithm \verb+make_hash+.
% \item \verb+c+ points to the initial state of the hashing automaton.
% \item \verb+i+ is a strictly positive integer.
% \item There is enough space to hold all of the elements being copied.
% \end{itemize}
% \paragraph{Complexity \\}
% Linear in the length of the output word.
% \paragraph{Example}
% \begin{verbatim}
% \end{verbatim}
% \paragraph{See Also \\}
% \verb+make_hash+, \verb+hash_value+
% %%%%%%%%%%%%%%%%%%%%%%%%%%% FIRST_MATCH %%%%%%%%%%%%%%%%%%%%%%%%%%%%%%%%%%%%%%%%%%%%%
% \newpage
% \noindent {\huge {\bf first\_match}} \\
% \noindent \begin{tabular}{p{7.25cm}p{7.25cm}} \hline
% \flushleft {\bf Category : {\Large Algorithm}} & \flushright {\bf Component Type :
% {\Large Function}} 
% \end{tabular} \\
% \paragraph{Prototype}
% \begin{verbatim}
% template <typename ForwardIterator, typename Cursor>
% ForwardIterator 
% first_match(ForwardIterator first, ForwardIterator last, Cursor &c);
% \end{verbatim}
% \paragraph{Description \\}
% Returns a past-the-end iterator on the shortest prefix of the word 
% \verb+[first, last)+ recognized by the automaton pointed by
% \verb+c+. If none is found, \verb+first+ is returned which means that
% either no final state has been reached during the traversal or a
% transition was undefined.
% \paragraph{Definition \\}
% \verb+match.h+
% \paragraph{Requirements on types}
% \begin{itemize}
% \item \verb+ForwardIterator+ is a model of forward iterator.
% \item \verb+Cursor+ is a model of plain cursor.
% \end{itemize}
% \paragraph{Preconditions}
% \begin{itemize}
% \item \verb+[first, last)+ is a valid range.
% \end{itemize}
% \paragraph{Complexity \\}
% Linear: at most \verb+last - first+ calls to \verb+Cursor::forward+.
% \paragraph{Example}
% \begin{verbatim}
% \end{verbatim}
% \paragraph{See Also \\}
% \verb+longest_match+, \verb+match_count+
% %%%%%%%%%%%%%%%%%%%%%%%% LONGEST_MATCH %%%%%%%%%%%%%%%%%%%%%%%%%%%%%%%%%%%%%%%%%%%%%%%%
% \newpage
% \noindent {\huge {\bf longest\_match}} \\
% \noindent \begin{tabular}{p{7.25cm}p{7.25cm}} \hline
% \flushleft {\bf Category : {\Large Algorithm}} & \flushright {\bf Component Type :
% {\Large Function}} 
% \end{tabular} \\
% \paragraph{Prototype}
% \begin{verbatim}
% template <class ForwardIterator, class Cursor>
% ForwardIterator 
% longest_match(ForwardIterator first, ForwardIterator last, Cursor &c);
% \end{verbatim}
% \paragraph{Description \\} 
% Returns a past-the-end iterator on the longest prefix of the word 
% \verb+[first, last)+ recognized by the automaton pointed by
% \verb+c+. If none is found, \verb+first+ is returned which means that
% either no final state has been reached during the traversal or a
% transition was undefined.
% \paragraph{Definition \\}
% \verb+match.h+
% \paragraph{Requirements on types}
% \begin{itemize}
% \item \verb+ForwardIterator+ is a model of forward iterator.
% \item \verb+Cursor+ is a model of plain cursor.
% \end{itemize}
% \paragraph{Preconditions}
% \begin{itemize}
% \item \verb+[first, last)+ is a valid range.
% \end{itemize}
% \paragraph{Complexity \\}
% Linear: at most \verb+last - first+ calls to \verb+Cursor::forward+.
% \paragraph{Example}
% \begin{verbatim}
% \end{verbatim}
% \paragraph{See Also \\}
% \verb+first_match+, \verb+match_count+
% %%%%%%%%%%%%%%%%%%%%%%%%% LANGUAGE %%%%%%%%%%%%%%%%%%%%%%%%%%%%%%%%%%%%%%%%%%%%%%%
% \newpage
% \noindent {\huge {\bf language}} \\
% \noindent \begin{tabular}{p{7.25cm}p{7.25cm}} \hline
% \flushleft {\bf Category : {\Large Algorithm}} & \flushright {\bf Component Type :
% {\Large Function}} 
% \end{tabular} \\
% \paragraph{Prototype}
% \begin{verbatim}
% template <typename DFirstCursor>
% void language(ostream &out,                              
%               DFirstCursor first, DFirstCursor last = DFirstCursor());
% \end{verbatim}
% \paragraph{Description \\}
% Output to \verb+out+ the language recognized by the DFA defined by
% the range \verb+[first, last)+.
% \paragraph{Definition \\}
% \verb+language.h+
% \paragraph{Requirements on types}
% \begin{itemize}
% \item \verb+DFirstCursor+ is a model of depth-first cursor.
% \item \verb+operator<<(ostream&, DFirstCursor::char_type)+ is
%   defined. 
% \end{itemize}
% \paragraph{Preconditions}
% \begin{itemize}
% \item \verb+[first, last)+ is a valid range.
% \end{itemize}
% \paragraph{Complexity \\}
% Linear
% \paragraph{Example}
% \begin{verbatim}
% \end{verbatim}
% \paragraph{See Also \\}
% \verb+language_iterator+
% %%%%%%%%%%%%%%%%%%%%%%%%%%% IS_IN %%%%%%%%%%%%%%%%%%%%%%%%%%%%%%%%%%%%%%%%%%%%%
% \newpage
% \noindent {\huge {\bf is\_in}} \\
% \noindent \begin{tabular}{p{7.25cm}p{7.25cm}} \hline
% \flushleft {\bf Category : {\Large Algorithm}} & \flushright {\bf Component Type :
% {\Large Function}} 
% \end{tabular} \\
% \paragraph{Prototype}
% \begin{verbatim}
% template <typename InputIterator, typename Cursor>
% bool is_in(InputIterator first, InputIterator last, Cursor c);
% \end{verbatim}
% \paragraph{Description \\}
% Returns \verb+true+ if the word \verb+[first, last)+ is in the
% language recognized by the DFA pointed by \verb+c+.
% \paragraph{Definition \\}
% \verb+language.h+
% \paragraph{Requirements on types}
% \begin{itemize}
% \item \verb+InputIterator+ is a model of input iterator.
% \item \verb+Cursor+ is a model of plain cursor.
% \end{itemize}
% \paragraph{Preconditions}
% \begin{itemize}
% \item \verb+[first, last)+ is a valid range.
% \end{itemize}
% \paragraph{Complexity \\}
% Linear: at most \verb+last - first+ calls to \verb+Cursor::forward+.
% \paragraph{Example}
% \begin{verbatim}
% \end{verbatim}
% \paragraph{See Also \\}
% \verb+first_match+, \verb+longest_match+
% %%%%%%%%%%%%%%%%%%%%%%%%%%% MATCH_COUNT %%%%%%%%%%%%%%%%%%%%%%%%%%%%%%%%%%%%%%%%%%%%%
% \newpage
% \noindent {\huge {\bf match\_count}} \\
% \noindent \begin{tabular}{p{7.25cm}p{7.25cm}} \hline
% \flushleft {\bf Category : {\Large Algorithm}} & \flushright {\bf Component Type :
% {\Large Function}} 
% \end{tabular} \\
% \paragraph{Prototype}
% \begin{verbatim}
% template <typename InputIterator, typename Cursor>
% int match_count(InputIterator first, InputIterator last, Cursor &c);
% \end{verbatim}
% \paragraph{Description \\} 
% Returns the number of prefixes of the word \verb+[first, last)+
% recognized by the DFA pointed to by \verb+c+, in other words, the
% number of final states \verb+c+ has reached along the path labelled
% with \verb+[first, last)+.
% \paragraph{Definition \\}
% \verb+match.h+
% \paragraph{Requirements on types}
% \begin{itemize}
% \item \verb+InputIterator+ is a model of input iterator.
% \item \verb+Cursor+ is a model of plain cursor.
% \end{itemize}
% \paragraph{Preconditions}
% \begin{itemize}
% \item \verb+[first, last)+ is a valid range.
% \end{itemize}
% \paragraph{Complexity \\}
% Linear: at most \verb+last - first+ calls to \verb+Cursor::forward+.
% \paragraph{Example}
% \begin{verbatim}
% \end{verbatim}
% \paragraph{See Also \\}
% \verb+first_match+, \verb+longest_match+
% %%%%%%%%%%%%%%%%%%%%%%%%%%%%% ISOMORPH %%%%%%%%%%%%%%%%%%%%%%%%%%%%%%%%%%%%%%%%%%%
% \newpage
% \noindent {\huge {\bf isomorph}} \\
% \noindent \begin{tabular}{p{7.25cm}p{7.25cm}} \hline
% \flushleft {\bf Category : {\Large Algorithm}} & \flushright {\bf Component Type :
% {\Large Function}} 
% \end{tabular} \\
% \paragraph{Prototype}
% \begin{verbatim}
% template <typename DFirstCursor1, typename DFirstCursor2>
% bool isomorph(DFirstCursor1 left, DFirstCursor2 first, 
%               DFirstCursor2 last = DFirstCursor2());
% \end{verbatim}
% \paragraph{Description \\} 
% Returns \verb+true+ if the DFA defined by the range 
% \verb+[left, DFirstCursor1())+ is isomorphic to the DFA defined by
% \verb+[first, last)+. 
% \paragraph{Definition \\}
% \verb+set_operation.h+
% \paragraph{Requirements on types}
% \begin{itemize}
% \item \verb+DFirstCursor1+ is a model of depth-first cursor.
% \item \verb+DFirstCursor2+ is a model of depth-first cursor.
% \item \verb+DFirstCursor2::char_type+ is convertible to
%   \verb+DFirstCursor1::char_type+.
% \end{itemize}
% \paragraph{Preconditions}
% \begin{itemize}
% \item \verb+[left, DFirstCursor1())+ is a valid range.
% \item \verb+[first, last)+ is a valid range.
% \end{itemize}
% \paragraph{Complexity \\}
% Linear
% \paragraph{Example}
% \begin{verbatim}
% \end{verbatim}
% %%%%%%%%%%%%%%%%%%%%%%%%%%%%% TREE_BUILD %%%%%%%%%%%%%%%%%%%%%%%%%%%%%%%%%%%%%%%%%%%
% \newpage
% \noindent {\huge {\bf add\_word, add\_words}} \\
% \noindent \begin{tabular}{p{7.25cm}p{7.25cm}} \hline
% \flushleft {\bf Category : {\Large Algorithm}} & \flushright {\bf Component Type :
% {\Large Function}} 
% \end{tabular} \\
% \paragraph{Prototype}
% \begin{verbatim}
% template <typename DFA1, typename InputIterator1>
% void add_word(DFA1 &a, InputIterator1 first, InputIterator1 last, 
%               const DFA1::tag_type &t = DFA1::tag_type());

% template <typename DFA2, typename InputIterator2>
% void add_words(DFA2 &a, InputIterator2 start, InputIterator2 finish);
% \end{verbatim}
% \paragraph{Description \\}
% \verb+add_word+ adds the word \verb+[first, last)+ to the language
% recognized by \verb+a+. The tag \verb+t+ is assigned to the newly
% created final state. \verb+add_words+ adds a set of words to the
% language, which means \verb+start+ points to containers of
% characters. \verb+a+ must have a tree structure.
% \paragraph{Definition \\}
% \verb+astl_tree.h+
% \paragraph{Requirements on types}
% \begin{itemize}
% \item \verb+InputIterator1+ and \verb+InputIterator2+ are models of
%   input iterator.
% \item \verb+DFA1+ and \verb+DFA2+ are models of DFA.
% \end{itemize}
% \paragraph{Preconditions}
% \begin{itemize}
% \item \verb+a+ has a tree structure, it must not be cyclic nor have a
%   DAG (Directed Acyclic Graph) structure.
% \item \verb+[first, last)+ is a valid range.
% \item \verb+InputIterator1::value_type+ is convertible to
%   \verb+DFA1::char_type+.
% \item \verb+[start, finish)+ is a valid range.
% \item \verb+InputIterator2::value_type+ is a container whose
%   \verb+value_type+ is \verb+DFA2::char_type+.
% \end{itemize}
% \paragraph{Complexity \\}
% \verb+add_word+: \verb+last - first+ calls to \verb+DFA::delta1+ or
% \verb+DFA1::set_trans+. \\
% \verb+add_words+: \verb+finish - start+ calls to \verb+add_word+.
% \paragraph{Example}
% \begin{verbatim}
% \end{verbatim}
\section{Advanced Examples}
\end{document}
