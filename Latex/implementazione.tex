\chapter{Implementazione}
L'architettura software del progetto di questa tesi è composta da tre moduli principali:
\begin{enumerate}
\item compilatore del linguaggio di specifica;
\item macchina diagnostica greedy;
\item macchina diagnostica lazy.
\end{enumerate}
La specifica dell'istanza di un sistema attivo complesso di cui si vuole determinare le diagnosi è espressa tramite un linguaggio creato ad hoc, la cui grammatica è fornita nell'appendice \ref{bnf}.
Durante la lettura del file di specifica, vengono generate le classi opportune e vengono effettuati i controlli del caso. Se questa procedura di compilazione non contiene errori (lessicali, sintattici o semantici), viene effettuato il preprocessing offline del problema, durante il quale si creano gli automi relativi ai pattern space. Tutte le informazioni ottenute vengono salvate su appositi file binari, in modo che non sia necessario dover effettuare nuovamente la procedura in fase di lavoro online, ovvero quando viene avviata la macchina diagnostica, sia essa in modalità di risoluzione greedy o lazy. 
La procedura diagnostica legge i file generati da una precedente compilazione della specifica del sistema e genera le soluzioni candidate. 
Il linguaggio di programmazione adottato è il C++, che è stato scelto in quanto buon compromesso tra le esigenze di efficienza richieste dal problema e il livello di astrazione della programmazione ad oggetti.

\section{Librerie esterne}
\subsection{Grail+}

\subsection{ASTL}
La libreria ASTL(\emph{Automata Standard Template Library})\cite{web:astl} è stata utilizzata in questo lavoro di tesi principalmente per avere delle strutture dati idonee a contenere automi con diversi stati e transizioni differenti. 

\subsection{Graphviz}

\subsection{Boost Serialization}




\section{Linguaggio di specifica}
Il linguaggio di specifica dei SAC studiati in questo lavoro di tesi è fornito, tramite notazione BNF, in appendice \ref{bnf}. 
Il file di specifica è articolato in quattro parti principali:
\begin{itemize}
\item modelli dei componenti, a cui i componenti concreti del sistema si riferiscono;
\item modelli dei nodi, a cui i nodi concreti del sistema si riferiscono;
\item sistema, che istanzia i nodi e definisce i link responsabili dell'invio di pattern event tra di essi;
\item problema, che fornisce informazioni necessarie per il problema diagnostico, come le osservazioni locali dei nodi.
\end{itemize}
I viewer locali, i ruler locali e gli stati iniziali dei singoli nodi possono essere definiti, con eventuale riscrittura, a tre diversi livelli:
\begin{itemize}
\item nella definizione del modello del nodo;
\item nella dichiarazione di un nodo del sistema;
\item nella specifica di un nodo del problema.
\end{itemize}
Questo permette una certa facilità nell'includere modelli di componenti e nodi senza ogni volta essere costretti a dichiarare nuovamente informazioni di questo tipo, quando sono ridondanti.
Il file di specifica viene compilato seguendo le tradizionali fasi ideate per i compilatori di linguaggi di programmazione:
\begin{enumerate}
\item analisi lessicale, che legge il flusso di caratteri che costituiscono il codice di specifica e raggruppa i caratteri in sequenze significative dette lessemi, quali le parole chiave del linguaggio e gli identificatori;
\item analisi sintattica, che riceve i lessemi dall'analizzatore lessicale per creare una rappresentazione ad albero del codice di specifica, detto albero sintattico;
\item analisi semantica, la quale utilizza le informazioni raccolte nell'albero sintattico in modo da verificare la consistenza del codice di specifica.
\end{enumerate}
Si noti che in questo progetto è possibile articolare una specifica in più file, esplicitando delle dichiarazioni del tipo \verb|#include| per i file che si vogliono incorporare. 
Questo permette il riutilizzo di specifiche già dichiarate precedentemente, aumentando inoltre la leggibilità dei file di specifica.


\subsection{Analizzatore lessicale}
L'analizzatore lessicale (o scanner) è implementato mediante lo strumento \emph{Flex}, il quale permette di dichiarare espressioni regolari che descrivono i pattern per i token, ovvero coppie nome-attributo, ad esempio \verb|<id,breaker>|. La struttura di un programma Flex  è composta da tre parti principali:
\begin{enumerate}
\item dichiarazioni, che include variabili, costanti e definizioni regolari;
\item regole di traduzione, che specificano per ogni pattern dichiarato per mezzo di una espressione regolare le azioni da compiere, ovvero frammenti di codice;
\item funzioni ausiliarie, che possono essere richiamate nelle azioni dei pattern.
\end{enumerate}
Sebbene Flex abbia un'opzione per creare direttamente uno scanner C++, il codice generato spesso non funziona e contiene errori\cite{book:bison} .
La procedura adottata per risolvere questo problema è quella di compilare un programma Flex scritto in linguaggio C per mezzo di un compilatore C++, facendo richiamare poi ad un analizzatore sintattico C++ lo scanner C definito.
La sezione dichiarativa del linguaggio è riportata di seguito. Essa definisce spazi, delimitatori, commenti di linea in stile C e l'espressione regolare che definisce gli identificatori \verb|id|, formata da una lettera a cui seguono altre lettere, numeri o  il carattere underscore.

\begin{verbatim}
// The location of the current token.
yy::location loc;

%}
%option noyywrap nounput
%x incl

delimiter	[ \t]
spacing		{delimiter}+
letter		[A-Za-z]
digit		[0-9]
unscore         _
id		{letter}({letter}|{digit}|{unscore})*
eol		\n
comment		\/\/.*

%{
  // Code run each time a pattern is matched.
  # define YY_USER_ACTION  loc.columns (yyleng);
%}
\end{verbatim}

Le regole di traduzione utilizzate sono principalmente volte a generare i token relativi a parole chiave del linguaggio e gli identificatori.
In corrispondenza di direttive \verb|#include|, il passaggio di lettura al file incluso è operato, come mostrato di seguito, creando una pila di buffer. 
\begin{verbatim}
#include          BEGIN(incl);
<incl>[ \t]*      /* eat the whitespace */
<incl>[^ \t\n]+   { /* got the include file name */
                 chdir(INPUT_FILE_DIR);
                 FILE *file = fopen( yytext, "r" );

                 if (file == NULL)
                 {
                     cout << "Input file error: included file does not exist" << endl;
                     exit(1);
                 }
                 yyin = file;

                 yypush_buffer_state(yy_create_buffer( yyin, YY_BUF_SIZE ));
                 BEGIN(INITIAL);
                 }
\end{verbatim}

Gli spazi e i commenti non invocano alcuna azione, mentre la fine del file corrente toglie dalla pila il buffer corrente, riportando il controllo alla scansione del file precedente da cui la direttiva di inclusione è stata chiamata. Se il buffer corrente è l'unico elemento della pila, il processo di scanning termina ed è generato il token relativo alla fine del file.
\begin{verbatim}
{spacing}	;
{comment}       ;

{eol}		{loc.lines (yyleng);loc.step();}

<<EOF>>    
{
  yypop_buffer_state();
  if ( !YY_CURRENT_BUFFER )
    return yy::spec_parser::make_END_OF_FILE(loc);;
  
}
\end{verbatim}
Quando viene trovata una parola chiave (come ad esempio \verb|component| e \verb|model|), viene creato il corrispondente token che verrà utilizzato nell'analisi sintattica.
Per quanto riguarda gli identificatori (\verb|id|), il loro token è composto oltre che dalla posizione, dalla stringa che ne rappresenta il nome, memorizzato durante lo scanning nella variabile \verb|yytext|.
\begin{verbatim}
component   return yy::spec_parser::make_COMPONENT(loc);
model       return yy::spec_parser::make_MODEL(loc);
is          return yy::spec_parser::make_IS(loc);


{id}        return yy::spec_parser::make_ID(yytext,loc);
.		{cout << "Lexical error: token \"" << yytext << "\" at " << loc << endl; exit(1);}
\end{verbatim}


\subsection{Analizzatore sintattico}
L'analizzatore sintattico (parser) è implementato tramite lo strumento \emph{Bison}, che ha permesso di utilizzare direttamente il linguaggio C++. L'utilizzo del linguaggio C++ è possibile inserendo all'inizio del file il comando \verb|%skeleton "lalr1.cc"|, il quale invoca il parser skeleton deterministico predisposto.




\subsection{Analizzatore semantico}


\subsection{Generazione dei pattern space}
La principale azione compiuta in fase offline, ovvero in un momento antecedente la diagnosi, è la generazione dei pattern space dei modelli di nodi definiti, come descritto nel paragrafo \ref{pts_construction}. 
Il primo passo di questo algoritmo consiste nella costruzione di un automa equivalente all'espressione regolare definita nella dichiarazione di pattern. Un metodo noto consiste nella costruzione di Thompson descritta nel paragrafo \ref{thompson}. 
Per questo scopo, è stata utilizzata la libreria \emph{Grail+}\cite{web:grail}, una libreria C++ che permette di definire espressioni regolari e convertirle in automi con un linguaggio equivalente.
La conversione avviene utilizzando la seguente sintassi:
\begin{verbatim}
fm<int> patodfa(std::string regex)
{
    re<int> r;
    istringstream str(regex);
    str >> r;
    fm<int> dfa;
    r.retofm(dfa);
    dfa.subset();

    return dfa;
}
\end{verbatim}
Data una stringa in ingresso, la funzione implementata inserisce tale stringa in un oggetto della classe \verb|re|(\emph{regular expression}), parametrizzato per mezzo di un template che indica l'alfabeto di appartenenza dei simboli dell'espressione. 
Dato che gli alfabeti possibili per questa libreria erano esclusivamente di tipo \verb|char| o \verb|int|, è stato scelto quest'ultimo (poiché 256 simboli possibili potevano non essere sufficienti in applicazioni di dimensioni ragguardevoli), mantenendo una corrispondenza biunivoca tra ogni intero e una transizione, coinvolta nella dichiarazione del pattern, relativa ad un componente del nodo. La conversione da espressione regolare ad automa a stati finiti avviene applicando all'oggetto che rappresenta l'espressione regolare il metodo \verb|retofm|, che crea l'automa nell'oggetto della classe \verb|fm|(\emph{finite machine}) passato come parametro. A seguito della conversione, l'automa è determinizzato applicando il metodo \verb|subset|, il quale esegue la subset construction descritta nel paragrafo\ref{subset}.
Successive manipolazioni dei diversi automi ottenuti dai pattern dichiarati permettono di unirli, aggiungere le $\epsilon$-transizioni opportune e determinizzare il risultato. L'automa finale viene poi convertito in un automa equivalente, utilizzando le strutture dati della libreria ASTL. Infatti, la libreria Grail+, sebbene utile poiché implementa la costruzione di Thompson, non sembrava indicata ad un utilizzo negli algoritmi di diagnosi, in quanto non permette estensioni per quanto riguarda i dati satellite associati a transizioni e stati dell'automa.
La minimizzazione finale del pattern space è implementata nel metodo \verb|Utils::minimize_by_partition|, che esegue una minimizzazione tenendo conto dei tag associati ai nodi finali dell'automa.

\subsection{Classi}


\section{Macchina diagnostica greedy}

\subsection{Classi}

\section{Macchina diagnostica lazy}

\subsection{Classi}