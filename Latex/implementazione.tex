\chapter{Implementazione}
L'architettura software del progetto di questa tesi è composta da tre moduli principali:
\begin{enumerate}
\item linguaggio di specifica;
\item diagnosi monolitica;
\item diagnosi distribuita.
\end{enumerate}
La specifica del particolare sistema attivo complesso di cui si vuole determinare la diagnosi è fornita tramite un linguaggio, la cui grammatica è fornita nell'appendice \ref{bnf}.
Durante il parsing del file di specifica, vengono generate le classi opportune e vengono effettuati i controlli semantici del caso. Se questa procedura non contiene errori (lessicali, sintattici o semantici), viene effettuato l'intero calcolo off-line del problema, durante il quale si generano le classi, e gli automi relativi ai pattern space. Le informazioni ottenute vengono salvate su dei file binari, in modo che non sia necessario dover effettuare nuovamente la procedura in fase di lavoro on-line, ovvero quando viene avviata la macchina diagnostica, sia essa in modalità di risoluzione monolitica o distribuita. 
La procedura diagnostica legge i file generati da una precedente compilazione della specifica del sistema e genera le soluzioni candidate. 

\section{Linguaggio di specifica}
Il linguaggio di specifica dei SAC studiati in questo lavoro di tesi è fornito, tramite notazione BNF, in appendice \ref{bnf}. 
Il file di specifica deve essere articolato in quattro parti principali:
\begin{itemize}
\item modelli dei componenti;
\item modelli dei nodi;
\item sistema;
\item problema.
\end{itemize}


\begin{figure}[htbp]
\begin{verbatim}
component model Breaker is
	event op, cl;
	input I;
	state closed, open;
	transition b1 = op(I), closed -> open, {},
				b2 = cl(I), open -> closed, {},
				b3 = op(I), closed -> closed, {},
				b4 = cl(I), open -> open, {},
				b5 = cl(I), closed -> closed, {},
				b6 = op(I), open -> open, {};
end Breaker.
\end{verbatim}
\caption{Specifica relativa al modello del componente Breaker}
\label{fig:spec_breaker}
\end{figure}


\begin{figure}[htbp]
\begin{verbatim}
component model ProtectionDevice is
	event op, cl;
	output O;
	state idle, awaken;
	transition p1 = (), idle -> awaken, {op(O)},
				p2 = (), awaken -> idle, {cl(O)},
				p3 = (), idle -> awaken, {cl(O)},
				p4 = (), awaken -> idle, {op(O)};
end ProtectionDevice.
\end{verbatim}
\caption{Specifica relativa al modello del componente Protection Device}
\label{fig:spec_protection}
\end{figure}



\subsection{Analizzatore lessicale}

\subsection{Analizzatore sintattico}

\subsection{Analizzatore semantico}

\section{Macchina diagnostica monolitica}

\section{Macchina diagnostica distribuita}