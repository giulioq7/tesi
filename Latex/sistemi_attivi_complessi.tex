\chapter{Sistemi Attivi Complessi}
In questo capitolo vengono presentate definizioni e modelli che descrivono i sistemi attivi complessi, visti come estensione dei sistemi attivi tradizionali. Un sistema attivo complesso è una rete di nodi, ognuno dei quali rappresenta un sistema attivo. Un nodo è dotato di terminali di input e di terminali di output; la rete del sistema è composta da link che collega terminali di output di un sistema attivo con uno o più terminali di input di altri nodi.


\section{Definizioni}
\subsection{Nodi}

\subsection{Pattern}
In modo da individuare pattern event, deve essere mantenuto lo stato relativo al matching dei pattern coinvolti. A tal fine, per ogni nodo, è necessario effettuare i seguenti passi:
\begin{itemize}
\item per ogni pattern $(p,r)$, viene generato un automa deterministico equivalente all'espressione regolare $r$, nel quale ogni stato finale viene marcato dal nome $p$ del pattern event;
\item per ogni linguaggio utilizzato nella definizione dei pattern, supponendo che sia condiviso da $k$ definizioni pattern corrispondenti agli automi $[A_1, \ldots , A_k]$ viene generato un pattern space, nel seguente modo:
	\begin{enumerate}
	\item un automa non deterministico $N$ è creato generando il suo stato iniziale $S_0$ e una $\epsilon$-			transition da $S_0$ a ogni stato iniziale di $A_i$, con $i \in [1 \ldots k]$;
	\item in modo da mantenere il matching di stringhe sovrapposte, viene aggiunta una $\epsilon$-transition
	da ogni stato non iniziale a $S_0$;
	\item l'automa $N$ viene determinizzato nel risultante pattern space $Pts(L)$, dove ogni stato finale $S$ è marcato dall'unione $p$ dei pattern event associati agli stati in $S$ che sono finali nei corrispondenti automi di pattern generati inizialmente. Ogni stato $S$ dell'automa deterministico $Pts(L)$ è infatti identificato da un sottoinsieme di stati dell'automa non deterministico equivalente $N$.
	\end{enumerate}
\end{itemize}

\begin{ex}
Viene di seguito presentata la costruzione del pattern space relativo a due pattern (caratterizzati ovviamente dallo stesso linguaggio).
\end{ex}

\subsection{Link della rete}

\subsection{Sistema attivo complesso}

\subsection{Traiettoria}

\subsection{Behavior space}

\section{Problema di diagnosi}
\subsection{Viewer}

\subsection{Osservazione temporale}

\subsection{Ruler}

\subsection{Problema di diagnosi}

\section{Diagnosi monolitica}
\subsection{Ricostruzione del behavior}

\subsection{Decorazione}

\subsection{Distillazione delle diagnosi}


\section{Diagnosi distribuita}
\subsection{Costruzione del Behavior non vincolato}

\subsection{Generazione dell'interfaccia}

\subsection{Costruzione del Behavior vincolato}

\subsection{Decorazione del Behavior del nodo radice}