\chapter{Introduzione}
Il lavoro di questa tesi si focalizza sulla diagnosi, la cui esecuzione automatica costituisce un ramo di forte interesse nell'ambito dell'intelligenza artificiale. Data un'osservazione riguardante il comportamento corrente di un sistema fisico, il ragionamento diagnostico permette di trovare eventuali guasti in tale comportamento. Questo, tipicamente, avviene mettendo in evidenza le discrepanze tra il comportamento normale e quello osservato, isolando le cause di tali differenze, chiamati guasti. L'espressione diagnosi basata sul modello si riferisce a tutti gli approcci di diagnosi nei quali vi è un modello esplicito che permette di inferire come il sistema debba nominalmente comportarsi. 
Per ragionare riguardo i sistemi dinamici, è spesso più efficiente modellare alcuni dei loro aspetti in termini discreti. I sistemi a eventi discreti sono sistemi dinamici con input e output discreti, il cui comportamento può essere descritto come cambiamenti di stato discreti. Molti sistemi dinamici reali possono infatti, ad un certo livello di astrazione, essere visti come sistemi a eventi discreti. Mentre nei modelli temporali viene considerato l'istante in cui si verificano input e output, i modelli privi del tempo (asincroni) tengono unicamente conto dell'ordine di occorrenza di tali eventi. 
In letteratura, vi sono stati approcci alla diagnosi sia mediante le reti di Petri, sia per mezzo degli automi. L'ultima scelta è quella più usata nell'ambito dell'intelligenza artificiale.
Un sistema attivo è un sistema dinamico modellato come un sistema fisico a eventi discreti distribuito, in cui alcuni ingressi non sono osservabili, in quanto provenienti dal mondo esterno e non facenti parte del sistema corrente. Un sistema attivo può essere visto come una rete di componenti interconnessi, i quali rimangono inattivi fintantoché non si verifica un particolare evento in ingresso. Una volta che si verifica un determinato evento, il componente effettua una transizione di stato, generando degli eventi in uscita, i quali possono essere catturati dai componenti collegati. A loro volta questi ultimi possono consumare eventi e generarne di altri, proseguendo in questo modo il meccanismo. Tale sequenza asincrona di transizioni di stato è chiamata reazione del sistema. Quando tale reazione termina, il sistema raggiunge uno stato quiescente, nel quale permane fino a che non interviene nuovamente un evento esterno al sistema a modificarne lo stato di quiete, provocando una nuova reazione. 
Un problema diagnostico derivato da un sistema attivo consiste da uno stato iniziale e da una osservazione del sistema. Effettuare la diagnosi di un sistema attivo richiede la ricostruzione di quanto succede a partire dallo stato iniziale, in base all'osservazione data. Questa procedura è detta ricostruzione della storia. Il modello necessita inoltre di informazioni che stabiliscano quali transizioni di stato siano normali e quali di guasto.
Vi sono tecniche risolutive che effettuano una diagnosi monolitica, di tutto il sistema, e tecniche che ne effettuano una decomposizione. Inoltre vi sono diversi approcci che ricercano un compromesso tra il calcolo off-line e on-line. 
Esistono diversi tipi di diagnosi: la diagnosi a posteriori, approfondita in questo lavoro, che consiste nell'operare una diagnosi a seguito di una intera sequenza di osservazione, a seguito della quale il sistema si porta in uno stato quiescente. Un altro approccio alla diagnosi è quello monitoring-based: la diagnosi viene effettuata passo a passo con il progredire dell'osservazione