\chapter{Introduzione}
Il lavoro di questa tesi si focalizza sulla diagnosi, la cui esecuzione automatica costituisce un ramo di forte interesse nell'ambito dell'intelligenza artificiale. Data un'osservazione riguardante il comportamento corrente di un sistema fisico, il ragionamento diagnostico permette di trovare eventuali guasti in tale comportamento. Questo, tipicamente, avviene mettendo in evidenza le discrepanze tra il comportamento normale e quello osservato, isolando le cause di tali differenze, chiamati guasti. Nello specifico, il lavoro svolto ha inizio dalle conoscenze del settore riguardante la diagnosi di sistemi attivi. Questi ultimi sono un particolare tipo di sistemi a eventi discreti, nel quale l'evoluzione dello stato è asincrona. Un sistema attivo si suppone formato da componenti che comunicano tra loro attraverso la consumazione e la generazione di eventi. La tesi svolta estende la specifica di un sistema attivo, come suggerito dalla fiorente ricerca degli ultimi anni, raggruppando i componenti interconnessi in sottosistemi, i quali conducono una vita propria eccetto ciò che riguarda il verificarsi di eventi particolari, scaturiti da determinate dinamiche interne. Un sistema attivo complesso è una rete di sistemi attivi, connessi tra loro in maniera gerarchica, formando un albero. Questa topologia di modello è suggerita in modo naturale da molti sistemi reali fisici, biologici, economici e sociali, nei quali l'evoluzione del sistema può essere vista a diversi livelli di astrazione. Si suppone che un sistema attivo possa influenzare il comportamento di un sistema attivo appartenente al livello immediatamente superiore nell'albero.
Sebbene l'applicazione originaria di sistemi siffatti nasca principalmente nell'ambito della diagnosi di reti elettriche, essi si prestano ad essere utilizzati in un qualsiasi contesto dal quale sia possibile estrapolare una stratificazione di comportamenti. Per citare un esempio, un sistema attivo complesso potrebbe in teoria costituire un modello per un sistema biologico: nel livello più basso della gerarchia vi sono le singole interazioni fra le cellule, ad un livello più alto i tessuti, aumentando il livello di astrazione vi è il funzionamento degli organi e degli apparati.
A fronte della definizione di questa nuova classe di sistemi, il lavoro svolto include una implementazione software riguardante la dichiarazione di problemi concreti di questo tipo per mezzo di un particolare linguaggio di specifica. Data una specifica, un compilatore, dopo aver effettuato i controlli necessari, genera le strutture dati atte a rappresentare l'istanza. L'elaborazione prosegue attraverso due metodi diagnostici risolutivi: un metodo greedy, che estende l'algoritmo noto nell'ambito dei sistemi attivi tradizionali a quello dei sistemi attivi complessi, calcolando l'evoluzione del sistema nella sua interezza, e un nuovo metodo lazy che, traendo vantaggio dalla topologia del sistema, calcola le diagnosi globali combinando le diagnosi di ogni singolo sistema attivo che compone la gerarchia con quelle delle interfacce dei nodi inferiori a esso collegati. Lo scopo di questa tesi è quello di mostrare, attraverso risultati sperimentali, la maggiore efficienza di questo nuovo metodo risolutivo.
La tesi è articolata nei capitoli seguenti:
\begin{itemize}
\item nel \emph{capitolo 2} è introdotto il concetto di automa a stati finiti, importante in questo lavoro sia nei modelli proposti che negli algoritmi risolutivi;
\item nel \emph{capitolo 3} è presentato lo stato dell'arte riguardante i sistemi attivi;
\item nel \emph{capitolo 4} sono fornite le definizioni e i metodi inerenti i sistemi attivi complessi;
\item nel \emph{capitolo 5} è descritta l'implementazione riferita alla specifica e al calcolo diagnostico,  nelle modalità greedy e lazy;
\item nel \emph{capitolo 6} sono forniti i risultati sperimentali che mostrano un confronto tra i due metodi e approfondiscono l'approccio lazy;
\item nel \emph{capitolo 7} sono inferite le conclusioni di questo lavoro e alcune considerazioni riguardanti i possibili sviluppi futuri;
\item l'\emph{appendice A} contiene la grammatica, in notazione BNF, del linguaggio di specifica;
\item l'\emph{appendice B} presenta un esempio di specifica di un sistema;
\item l'\emph{appendice C} fornisce le istruzione di installazione del software sviluppato.
\end{itemize}
