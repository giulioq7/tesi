\chapter{Sperimentazione}
La fase di sperimentazione è stata compiuta essenzialmente facendo un confronto tra i tempi di esecuzione e la memoria consumata dalla macchina diagnostica monolitica e da quella distribuita. I risultati confermano l'aspettativa del miglioramento in termini di efficienza a favore di quest'ultimo approccio. \'E stato altresì verificato, come atteso, che il metodo monolitico fosse applicabile a istanze piccole di problemi di diagnosi, e che la memoria potesse essere consumata interamente per pochi componenti. A seguito di tale verifica, sono stati condotti degli esperimenti, relativi alla sola macchina diagnostica distribuita, in modo da poter analizzare la complessità reale dell'algoritmo, aumentando il numero di componenti e di nodi del sistema. Il risultato è estremamente soddisfacente, in quanto ha permesso di catturare un andamento lineare sia in termini di tempo computazionale, sia in termini di memoria occupata.


\section{Confronto dei casi monolitico e distribuito}
\subsection{Tempo di esecuzione}

\subsection{Memoria}

\section{Approfondimento del caso distribuito}
